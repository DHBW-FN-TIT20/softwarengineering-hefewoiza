\section{Fachklassendiagramm}

\noindent Das folgende Fachklassendiagramm \autoref{fig:Fachklassendiagramm} zeigt die benötigten Klassen mit ihren
Kardinalitäten, Attributen und Operationen für die Umsetzung des \ac{aMRS}. Bei der Modellierung wurden Design Patterns
berücksichtigt, um eine Entkopplung der Komponenten zu verwirklichen. Durch die Entkopplung soll eine bessere
Wartbarkeit, Erweiterbarkeit und die Qualität der Software sichergestellt werden. \newline

\noindent Wie in \autoref{fig:Fachklassendiagramm} zu erkennen, ist das System in zwei unabhängige Teilsysteme
untergliedert, die nur über die Datenspeicherung verbunden sind. Diese nutzt das Repository Pattern, um eine
Entkopplung der Datenzugriffsschicht vom restlichen System zu ermöglichen. Über das Adapter Pattern ist eine
Schnittstelle zwischen \hyperref[gls:informationsportal]{Informationsportal} und \ac{aMRS} geplant. Zum andern bietet \ac{aMRS} mit ihrer eigenen GUI die
Möglichkeit den im \autoref{cha:AblaufbeschreibungderSystemidee} beschriebenen Bewertungsprozess durchzuführen. Dieser Prozess nutzt einen Adapter
zu einem externen \hyperref[gls:ocr-System]{OCR-System} und diverse Validatoren, die mithilfe des Iterator Pattern leicht an zukünftige
Kundenwünsche angepasst werden können.

\newpage

\begin{figure}[H]
    \centering
    \caption{Fachklassendiagramm} \label{fig:Fachklassendiagramm}
    \includegraphics[width=\textwidth,keepaspectratio]{images/Fachklassenmodell}
\end{figure}

\section*{Beschreibung der Fachklassen}

\subsection*{Entity}
\textbf{Dish} \\
Die Klasse Dish enthält Informationen über \hyperref[gls:restaurantname]{Restaurantname}, \hyperref[gls:restaurantAdresse]{Restaurantadresse}, \hyperref[gls:Rechnungsbezeichner]{Rechnungsbezeichner} und \hyperref[gls:gerichtname]{Gerichtname} des
\hyperref[gls:tagesgericht]{Tagesgerichts}.
\newline

\noindent \textbf{BillData} \\
Die Klasse BillData enthält Informationen über \hyperref[gls:restaurantname]{Restaurantname}, \hyperref[gls:restaurantAdresse]{Restaurantadresse}, Rechnungsnummer und eine Liste aller
\hyperref[gls:Rechnungsbezeichner]{Rechnungsbezeichner} der \hyperref[gls:restaurantRechnung]{Restaurantrechnung}.
\newline

\noindent \textbf{Rating} \\
Die Klasse Rating enthält Informationen über das zu bewertende Dish, eine Wertung und ein Kommentar.

\subsection*{Adapter}
\noindent \textbf{aMRSAdapter}\\
Die Klasse aMRSAdapter gibt dem \hyperref[gls:informationsportal]{Informationsportal} die Möglichkeit, Bewertungen vom \ac{aMRS} abzurufen.
\newline

\noindent \textbf{InformationsPortalAdapter}\\
Die Klasse InformationsPortalAdapter gibt dem \ac{aMRS} die Möglichkeit, \hyperref[gls:tagesgericht]{Tagesgerichte} vom \hyperref[gls:informationsportal]{Informationsportal} abzurufen.
Dabei werden die \hyperref[gls:tagesgericht]{Tagesgerichte} nach Klassifikation gefiltert.
\newline

\noindent \hyperref[gls:ocr-System]{\textbf{OCR-BillAnalyzer}}\\
Die Klasse \hyperref[gls:ocr-System]{OCR-BillAnalyzer} stellt die Kommunikation von \ac{aMRS} zum externen \hyperref[gls:ocr-System]{OCR-System} her.


\subsection*{Boundary}
\noindent \textbf{Connection}\\
Die Klasse Connection empfängt Befehle zur Datenspeicherung beziehungsweise zum Datenzugriff.

\subsection*{Klasse}
\noindent \textbf{DishRepository<Dish>}\\
Die Klasse DishRepository ist für die Speicherung und den Zugriff auf die Daten der Klasse Dish zuständig.
\newline

\noindent \textbf{RatingRepository<Rating>}\\
Die Klasse RatingRepository ist für die Speicherung und den Zugriff auf die Daten der Klasse Rating zuständig.
\newline

\noindent \textbf{DishManager}\\
Die Klasse DishManager enthält die Logik für das Aktualisieren und Speichern neuer Dish-Objekte, sowie zum Abrufen von
vorhandenen Bewertungen.
\newline

\noindent \textbf{RatingManager}\\
Die Klasse RatingManager enthält die Logik für das Speichern und Validieren neuer Rating-Objekte. Zusätzlich wird das
Objekt BillData auf Authentizität überprüft.
\newline

\noindent \textbf{ScamValidator}\\
Die Klasse ScamValidator ist für die Überprüfung auf Scam in Bewertungen zuständig.
\newline

\noindent \textbf{SpamValidator}\\
Die Klasse SpamValidator ist für die Überprüfung auf Spam in Bewertungen zuständig.
\newline

\noindent \textbf{HateValidator}\\
Die Klasse HateValidator ist für die Überprüfung auf \hyperref[gls:hass]{Hass} in Bewertungen zuständig.
\newline

\noindent \textbf{ValidatorIterator}\\
Die Klasse ValidatorIterator ist für die Iteration über die Validatoren zuständig.
\newline

\noindent \textbf{aMRSController}\\
Die Klasse aMRSController ist für die Kommunikation zwischen View und RatingManager zuständig.
\newline

\noindent \textbf{View}\\
Die Klasse View stellt die GUI dar.