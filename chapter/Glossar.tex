\section*{Glossar}\label{cha:glossar}
\addcontentsline{toc}{section}{Glossar}

% #TODO: Checken ob periodisch noch definiert werden muss

\begin{table}[H]
    \centering
    \label{gls:auftraggeber}
    \begin{tabularx}{\textwidth}{| l | X |}
        \hline
        Begriff         & Auftraggeber                                                                                                         \\
        \hline
        Synonyme        & Kunde, Abnehmer, Klient                                                                                              \\
        \hline
        Definition      & Person, Firma, Institution oder Ähnliches, die einen Auftrag erteilt. Dies ist der Betreiber des Infomrationsportal. \\
        \hline
        Abgrenzung      & -                                                                                                                    \\
        \hline
        Einschränkungen & Keine                                                                                                                \\
        \hline
        Ansprechpartner & Florian Glaser                                                                                                       \\
        \hline
        Status          & Entwurf                                                                                                              \\
        \hline
        Änderungen      & 08.12.2022: erstellt                                                                                                 \\
        \hline
    \end{tabularx}
\end{table}


\begin{table}[H]
    \centering
    \label{gls:authentischeBewertung}
    \begin{tabularx}{\textwidth}{| l | X |}
        \hline
        Begriff         & Authentische Bewertung                                                                                                                                                                                                                                                                        \\
        \hline
        Synonyme        & Seriöse Bewertung, vertrauenswürdige Bewertung, echte Bewertung, beglaubigte Bewertung, belegte Bewertung                                                                                                                                                                                     \\
        \hline
        Definition      & Eine Bewertung ist authentisch, sobald diese von einer Person verfasst wurde, welche verifiziert das Produkt gekauft hat. Eine Bewertung beinhaltet einen Text zusammen mit einer numerischen Bewertung zwischen eins und fünf. Dabei ist fünf die beste und eins die schlechteste Bewertung. \\
        \hline
        Abgrenzung      & Normale, unseriöse Bewertung: Die Bewertung kann ohne Verifizierung des Users erfolgen.                                                                                                                                                                                                       \\
        \hline
        Einschränkungen & Keine                                                                                                                                                                                                                                                                                         \\
        \hline
        Ansprechpartner & Florian Glaser                                                                                                                                                                                                                                                                                \\
        \hline
        Status          & Entwurf                                                                                                                                                                                                                                                                                       \\
        \hline
        Änderungen      & 08.12.2022: erstellt                                                                                                                                                                                                                                                                          \\
        \hline
    \end{tabularx}
\end{table}

\begin{table}[H]
    \centering
    \label{gls:gerichtname}
    \begin{tabularx}{\textwidth}{| l | X |}
        \hline
        Begriff         & Gerichtname                                                                                                                       \\
        \hline
        Synonyme        & Tagesgerichtname, Speisename, Essensname                                                                                          \\
        \hline
        Definition      & Eindeutiger Name eines Gerichts eines \hyperref[gls:restaurant]{Restaurants}, welcher auf einer Tageskarte und Speisekarte steht. \\
        \hline
        Abgrenzung      & -                                                                                                                                 \\
        \hline
        Einschränkungen & Keine                                                                                                                             \\
        \hline
        Ansprechpartner & Florian Herkommer                                                                                                                 \\
        \hline
        Status          & Entwurf                                                                                                                           \\
        \hline
        Änderungen      & 08.12.2022: erstellt                                                                                                              \\
        \hline
    \end{tabularx}
\end{table}



\begin{table}[H]
    \centering
    \label{gls:Rechnungsbezeichner}
    \begin{tabularx}{\textwidth}{| l | X |}
        \hline
        Begriff         & Rechnungsbezeichner                                                                                                                      \\
        \hline
        Synonyme        & Gerichtbezeichner, Tagesgerichtbezeichner, Speisebezeichner, Essensbezeichner                                                            \\
        \hline
        Definition      & Eindeutiger Name eines Gerichts eines \hyperref[gls:restaurant]{Restaurants}, welcher auf der Rechnung steht und aMRS bekannt sein muss. \\
        \hline
        Abgrenzung      & -                                                                                                                                        \\
        \hline
        Einschränkungen & Keine                                                                                                                                    \\
        \hline
        Ansprechpartner & Florian Herkommer                                                                                                                        \\
        \hline
        Status          & Entwurf                                                                                                                                  \\
        \hline
        Änderungen      & 16.12.2022: erstellt                                                                                                                     \\
        \hline
    \end{tabularx}
\end{table}

\begin{table}[H]
    \centering
    \label{gls:hass}
    \begin{tabularx}{\textwidth}{| l | X |}
        \hline
        Begriff         & Hass                                                                                                                                                               \\
        \hline
        Synonyme        & Hassrede, hate                                                                                                                                                     \\
        \hline
        Definition      & Hass bezeichnet innerhalb von aMRS sprachliche Ausdrucksweisen von Hass mit dem Ziel der Herabsetzung und Verunglimpfung bestimmter Personen oder Personengruppen. \\
        \hline
        Abgrenzung      & -                                                                                                                                                                  \\
        \hline
        Einschränkungen & Keine                                                                                                                                                              \\
        \hline
        Ansprechpartner & Florian Herkommer                                                                                                                                                  \\
        \hline
        Status          & Entwurf                                                                                                                                                            \\
        \hline
        Änderungen      & 16.12.2022: erstellt                                                                                                                                               \\
        \hline
    \end{tabularx}
\end{table}


\begin{table}[H]
    \centering
    \label{gls:informationsportal}
    \begin{tabularx}{\textwidth}{| l | X |}
        \hline
        Begriff         & Informationsportal                                                                                                                                                                                                                                                      \\
        \hline
        Synonyme        & Portal, Informationsschnittstelle                                                                                                                                                                                                                                       \\
        \hline
        Definition      & Ein Informationsportal bietet Informationen über aktuelle \hyperref[gls:tagesgericht]{Tagesgerichte} über eine Schnittstelle. Dieses Portal stellt im Sinne der OOA den \hyperref[gls:auftraggeber]{Auftraggeber} dar. Dieses Portal soll um das aMRS erweitert werden. \\
        \hline
        Abgrenzung      & Das Informationsportal ist nicht allgemein definiert, sondern nur für die Schnittstelle die Informationen zu \hyperref[gls:tagesgericht]{Tagesgerichten} anbietet.                                                                                                      \\
        \hline
        Einschränkungen & keine                                                                                                                                                                                                                                                                   \\
        \hline
        Ansprechpartner & Baldur Siegel                                                                                                                                                                                                                                                           \\
        \hline
        Status          & Entwurf                                                                                                                                                                                                                                                                 \\
        \hline
        Änderungen      & 08.12.2022: erstellt
        \newline 16.12.2022: überarbeitet                                                                                                                                                                                                                                                         \\
        \hline
    \end{tabularx}
\end{table}



\begin{table}[H]
    \centering
    \label{gls:nutzer}
    \begin{tabularx}{\textwidth}{| l | X |}
        \hline
        Begriff         & Nutzer                                                                                   \\
        \hline
        Synonyme        & Benutzer, Anwender, Bediener                                                             \\
        \hline
        Definition      & Ein Nutzer beschreibt eine Person, welches ein Produkt nutzt und mit diesem interagiert. \\
        \hline
        Abgrenzung      & -                                                                                        \\
        \hline
        Einschränkungen & keine                                                                                    \\
        \hline
        Ansprechpartner & Baldur Siegel                                                                            \\
        \hline
        Status          & Entwurf                                                                                  \\
        \hline
        Änderungen      & 08.12.2022: erstellt                                                                     \\
        \hline
    \end{tabularx}
\end{table}

\begin{table}[H]
    \centering
    \label{gls:ocr-System}
    \begin{tabularx}{\textwidth}{| l | X |}
        \hline
        Begriff         & OCR-System                                                                                                                                                                                                       \\
        \hline
        Synonyme        & OCR-BillAnalyzer, Texterkennung, optische Zeichenerkennung                                                                                                                                                                               \\
        \hline
        Definition      & Der OCR-BillAnalyzer bezeichnet die automatisierte Texterkennung beziehungsweise automatische Schrifterkennung innerhalb von Bildern und liefert die Daten ans aMRS zurück. Dieses System wird extern eingebunden \\
        \hline
        Abgrenzung      & -                                                                                                                                                                                                                 \\
        \hline
        Einschränkungen & Keine                                                                                                                                                                                                             \\
        \hline
        Ansprechpartner & Florian Glaser                                                                                                                                                                                                    \\
        \hline
        Status          & Entwurf                                                                                                                                                                                                           \\
        \hline
        Änderungen      & 16.12.2022: erstellt                                                                                                                                                                                              \\
        \hline
    \end{tabularx}
\end{table}


\begin{table}[H]
    \centering
    \label{gls:restaurant}
    \begin{tabularx}{\textwidth}{| l | X |}
        \hline
        Begriff         & Restaurant                              \\
        \hline
        Synonyme        & Lokal, Gasthof, Gasthaus                \\
        \hline
        Definition      & Gaststätte, in der Essen serviert wird. \\
        \hline
        Abgrenzung      & -                                       \\
        \hline
        Einschränkungen & keine                                   \\
        \hline
        Ansprechpartner & Baldur Siegel                           \\
        \hline
        Status          & Entwurf                                 \\
        \hline
        Änderungen      & 08.12.2022: erstellt                    \\
        \hline
    \end{tabularx}
\end{table}

\begin{table}[H]
    \centering
    \label{gls:restaurantname}
    \begin{tabularx}{\textwidth}{| l | X |}
        \hline
        Begriff         & Restaurantname                                                                                             \\
        \hline
        Synonyme        & Lokal, Lokalname, Restaurant                                                                               \\
        \hline
        Definition      & Vollständiger Name des Restaurants, in dem die \hyperref[gls:tagesgericht]{Tagesgerichte} verkauft werden. \\
        \hline
        Abgrenzung      & -                                                                                                          \\
        \hline
        Einschränkungen & Keine                                                                                                      \\
        \hline
        Ansprechpartner & Florian Glaser                                                                                             \\
        \hline
        Status          & Entwurf                                                                                                    \\
        \hline
        Änderungen      & 08.12.2022: erstellt                                                                                       \\
        \hline
    \end{tabularx}
\end{table}

\begin{table}[H]
    \centering
    \label{gls:restaurantAdresse}
    \begin{tabularx}{\textwidth}{| l | X |}
        \hline
        Begriff         & Restaurantadresse                                                                                                                                          \\
        \hline
        Synonyme        & Restaurantanschrift                                                                                                                                        \\
        \hline
        Definition      & Die Restaurantadresse beschreibt die vollständige Anschrift eines Restaurants. Diese beinhaltet Straße, Stadt, Postleitzahl, Adresszusatz, Land, Landkreis \\
        \hline
        Abgrenzung      & -                                                                                                                                                          \\
        \hline
        Einschränkungen & Keine                                                                                                                                                      \\
        \hline
        Ansprechpartner & Phillipp Patzelt                                                                                                                                           \\
        \hline
        Status          & Entwurf                                                                                                                                                    \\
        \hline
        Änderungen      & 08.12.22: erstellt                                                                                                                                         \\
        \hline
    \end{tabularx}
\end{table}

\begin{table}[H]
    \centering
    \label{gls:restaurantRechnung}
    \begin{tabularx}{\textwidth}{| l | X |}
        \hline
        Begriff         & Restaurantrechnung                                                                                                                                                                                                                                                                                                                                \\
        \hline
        Synonyme        & Bon, Beleg, Kassenbon, Quittung                                                                                                                                                                                                                                                                                                                   \\
        \hline
        Definition      & Die Restaurantrechnung beschreibt einen Kassenbon, der am Ende einer Zahlung dem Kunden ausgestellt wird. Dieser wird im aMRS auch zur Verifizierung einer authentischen Bewertung genutzt. Damit dieser rechtens ausgestellt werden kann, muss er folgende Pflichtangaben aufweisen \autocite{bernhard_kostler_kassenbon-pflicht_2022}:
        \begin{itemize}
            \item Der vollständige Name sowie die Anschrift des Unternehmens
            \item Das Datum der Ausstellung
            \item Der Zeitpunkt des Vorgangsbeginns sowie des Vorgangsendes
            \item Die Menge und die Art der Bestellung
            \item Die eindeutige Transaktionsnummer
            \item Der fällige Zahlbetrag und Steuerbetrag
            \item Seriennummer des elektronischen Kassensystems
        \end{itemize}
        \\
        \hline
        Abgrenzung      & -                                                                                                                                                                                                                                                                                                                                                 \\
        \hline
        Einschränkungen & Keine                                                                                                                                                                                                                                                                                                                                             \\
        \hline
        Ansprechpartner & Phillipp Patzelt                                                                                                                                                                                                                                                                                                                                  \\
        \hline
        Status          & Entwurf                                                                                                                                                                                                                                                                                                                                           \\
        \hline
        Änderungen      & 08.12.2022: erstellt                                                                                                                                                                                                                                                                                                                              \\
        \hline
    \end{tabularx}
\end{table}

\begin{table}[H]
    \centering
    \label{gls:tagesgericht}
    \begin{tabularx}{\textwidth}{| l | X |}
        \hline
        Begriff         & Tagesgericht                                                                                                                                                                                                                                                                                                                                                                                         \\
        \hline
        Synonyme        & Gericht, Tagesspeise, Tagesessen                                                                                                                                                                                                                                                                                                                                                                     \\
        \hline
        Definition      & Ein Tagesgericht beschreibt ein Gericht, welches in einem \hyperref[gls:restaurant]{Restaurant} an einem bestimmten Tag angeboten wird. Ein Tagesgericht wird auf einer Tageskarte ausgeschrieben und wird automatisiert in das \hyperref[gls:informationsportal]{Informationsportal} übertragen. Das Tagesgericht kann wiederkehrend (z.B.: jeden Montag) auf der Tageskarte ausgeschrieben werden. \\
        \hline
        Abgrenzung      & Als Tagesgericht werden nicht Gerichte eines \hyperref[gls:restaurant]{Restaurants} bezeichnet, welche im festen Angebot stehen.                                                                                                                                                                                                                                                                     \\
        \hline
        Einschränkungen & Keine                                                                                                                                                                                                                                                                                                                                                                                                \\
        \hline
        Ansprechpartner & Florian Herkommer                                                                                                                                                                                                                                                                                                                                                                                    \\
        \hline
        Status          & Enwturf                                                                                                                                                                                                                                                                                                                                                                                              \\
        \hline
        Änderungen      & 08.12.2022: erstellt                                                                                                                                                                                                                                                                                                                                                                                 \\
        \hline
    \end{tabularx}
\end{table}


\begin{table}[H]
    \centering
    \label{gls:vegan}
    \begin{tabularx}{\textwidth}{| l | X |}
        \hline
        Begriff         & Vegan                                                                                                                                                                                                                                                                                                                             \\
        \hline
        Synonyme        & Veganismus                                                                                                                                                                                                                                                                                                                        \\
        \hline
        Definition      & Vegan sind Lebensmittel, die keine Erzeugnisse tierischen Ursprungs sind und bei denen auf allen Produktions- und Verarbeitungsstufen keine Zutaten, Verarbeitungshilfsstoffe oder Nicht-Lebensmittelzusatzstoffe die tierischen Ursprungs sind, in verarbeiteter oder unverarbeiteter Form zugesetzt oder verwendet worden sind. \\
        \hline
        Abgrenzung      & -                                                                                                                                                                                                                                                                                                                                 \\
        \hline
        Einschränkungen & keine                                                                                                                                                                                                                                                                                                                             \\
        \hline
        Ansprechpartner & Baldur Siegel                                                                                                                                                                                                                                                                                                                     \\
        \hline
        Status          & Entwurf                                                                                                                                                                                                                                                                                                                           \\
        \hline
        Änderungen      & 08.12.2022: erstellt                                                                                                                                                                                                                                                                                                              \\
        \hline
    \end{tabularx}
\end{table}