\section{Ablaufbeschreibung der Systemidee}
% periodisch vom Informationsportal
% Dabei Prüfung auf (konfigurierbarer Filter) Vegan (Tag) und Abgleich eigene Datenbank -> nur neue Items
% DATEN: Name, Restaurant-Name, Restaurant-Adresse, Rechnungsbezeichner (identisch zu Name auf Rechnung), Tags (vegan)
Für die Betreibung von \ac{aMRS} werden die folgenden Daten Periodisch (z.B.: jede fünf Minuten) vom Informationsportal abgerufen und in \ac{aMRS} gespeichert:
Die zur eindeutigen Identifikation der Tagesgerichte benötigten Attribute Restaurantname, Restaurantadresse und Rechnungsbezeichner, sowie die Attribute Gerichtname und Klassifikationen (z.B.: Vegan).
Die bezogenen Daten werden in jeder Periode zunächst auf ausgewählte Klassifikationen (z.B.: Vegan) gefiltert und anschließend mit den bestehenden Daten in \ac{aMRS} verglichen, sodass das System ausschließlich um neue Datensätze erweitert wird.
Bei der Konfiguration von \ac{aMRS} können die zu filternden Klassifikationen konfiguriert werden, wodurch die durch den Auftraggeber geforderte Spezifizierung auf Vegane Gerichte ermöglicht wird.

% Übertragung der Daten / Bereitstellungsformat ist durch den Auftraggeber spezifizierbar, bzw. beliebig austauschbar
% -> Unser System kann die Daten über jedes Format entgegennehmen
Die Modularität von \ac{aMRS} ermöglicht einen Technologie unabhängigen Datenaustausch zwischen dem Informationsportal und \ac{aMRS}, wodurch die Bereitstellung der Daten durch das Informationsportal flexibel gestaltet ist.

% Ablauf Bewertungsprozess
Durch das Hochladen der Restaurantrechnung auf der von \ac{aMRS} bereitgestellten Benutzeroberfläche wird der Bewertungsprozess eingeleitet.
