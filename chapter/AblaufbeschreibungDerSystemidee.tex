\section{Ablaufbeschreibung der Systemidee}\label{cha:AblaufbeschreibungderSystemidee}
% periodisch vom \hyperref[gls:informationsportal]{Informationsportal}
% Dabei Prüfung auf (konfigurierbarer Filter) Vegan (Tag) und Abgleich eigene Datenbank -> nur neue Items
% DATEN: Name, Restaurant-Name, Restaurant-Adresse, Rechnungsbezeichner (identisch zu Name auf Rechnung), Tags (vegan)
% TODO: Evtl. in Kapitel Machbarkeit beschreiben, dass 5 Minuten valide sind, da Kunde erst das Gericht bestellen und essen muss bevor eine Rechnung ausgestellt und somit das Gericht bewertet werden kann.
Für die Betreibung von \ac{aMRS} werden die folgenden Daten Periodisch (z.B.: jede fünf Minuten) vom \hyperref[gls:informationsportal]{Informationsportal} abgerufen und in \ac{aMRS} persistent gespeichert:
Die zur eindeutigen Identifikation der Tagesgerichte benötigten Attribute \hyperref[gls:restaurantname]{Restaurantname}, \hyperref[gls:restaurantadresse]{Restaurantadresse} und \hyperref[gls:Rechnungsbezeichner]{Rechnungsbezeichner}, sowie die Attribute \hyperref[gls:gerichtname]{Gerichtname} und Klassifikationen (z.B.: Vegan).
Die bezogenen Daten werden in jeder Periode zunächst auf ausgewählte Klassifikationen (z.B.: Vegan) gefiltert und anschließend mit den bestehenden Daten in \ac{aMRS} verglichen, sodass das System ausschließlich um neue Datensätze erweitert wird.
Bei der Konfiguration von \ac{aMRS} können die zu filternden Klassifikationen konfiguriert werden, wodurch die durch den \hyperref[gls:auftraggeber]{Auftraggeber} geforderte Spezifizierung auf Vegane Gerichte ermöglicht wird.
\newparagraph
Im Gegensatz zu einem Eventbasierten Ansatz, bei dem die Daten nur bei einer Änderung des Infomationsportals abgerufen werden, erfordert die Periodische Abfrage der Daten keine umfangreiche Modifikation des \hyperref[gls:informationsportal]{Informationsportals} durch den \hyperref[gls:auftraggeber]{Auftraggeber}.
Während das \hyperref[gls:informationsportal]{Informationsportal} bei einem Eventbasierten Ansatz bei jeder Änderung der Daten ein Event an \ac{aMRS} senden muss, reicht es bei der Periodischen Abfrage der Daten aus, die Gesamtdaten des \hyperref[gls:informationsportal]{Informationsportals} für \ac{aMRS} zur Verfügung zu stellen.
\newparagraph
% Übertragung der Daten / Bereitstellungsformat ist durch den Auftraggeber spezifizierbar, bzw. beliebig austauschbar
% -> Unser System kann die Daten über jedes Format entgegennehmen
Die Modularität von \ac{aMRS} ermöglicht einen Technologie unabhängigen Datenaustausch zwischen dem \hyperref[gls:informationsportal]{Informationsportal} und \ac{aMRS}, wodurch die Bereitstellung der Daten durch das \hyperref[gls:informationsportal]{Informationsportal} flexibel gestaltet ist.
\newparagraph
% Ablauf Bewertungsprozess
Um den Bewertungsprozess auf der von \ac{aMRS} bereitgestellten Benutzeroberfläche zu starten, muss der Bewerter die \hyperref[gls:restaurantRechnung]{Restaurantrechnung} als Bild bereitstellen.
Dieses wird anschließend an ein externes System -- den Rechnungsscanner -- übergeben, welches den Inhalt der \hyperref[gls:restaurantRechnung]{Restaurantrechnung} zurück liefert.
Die erhaltenen Daten sind dabei bereits in die folgenden Attribute untergliedert:
\begin{itemize}
    \item Alle \hyperref[gls:Rechnungsbezeichner]{Rechnungsbezeichner} die auf der Rechnung aufgelistet sind
    \item \hyperref[gls:restaurantname]{Restaurantname}
    \item \hyperref[gls:restaurantadresse]{Restaurantadresse}
    \item Rechnungsnummer
\end{itemize}
Ausgehend von den erhaltenen Daten gleicht \ac{aMRS} die enthaltenen Gerichte mit den im System gespeicherten Daten ab.
Dabei werden die Gerichte, welche bereits mit der Rechnungsnummer bewertet wurden, ebenfalls durch \ac{aMRS} identifiziert.
Somit kann \ac{aMRS} ermitteln, welche Gerichte für den Bewerter zur Bewertung freigegeben werden.
\ac{aMRS} leitet diese Information an die Benutzeroberfläche weiter, wodurch die Eingabe der Bewertung startet.
\newparagraph
Durch das Absenden der Bewertung auf der Benutzeroberfläche wird die Bewertung zunächst durch \ac{aMRS} auf dessen Inhalt überprüft.
% INFO: Inhaltsprüfung nicht extern -> Wir definieren wie die Überprüfung aussieht, nicht modular für den Kunden -> Wenn Kunde spezielle Anforderungen hat, dann muss er das in seinen Requirements definieren 
Entspricht die erfasste Bewertung den festgelegten Kriterien, wird diese in \ac{aMRS} persistent gespeichert.
\newparagraph
Über eine separate Schnittstelle kann das \hyperref[gls:informationsportal]{Informationsportal} die Bewertungen zu den Gerichten abrufen.
% INFO: Die Bewertung besteht aus Text und Zahl (1-5)
Die durch \ac{aMRS} bereitgestellten Daten enthalten dabei die Attribute \hyperref[gls:restaurantname]{Restaurantname}, \hyperref[gls:restaurantadresse]{Restaurantadresse}, \hyperref[gls:Rechnungsbezeichner]{Rechnungsbezeichner} und die Bewertung.
