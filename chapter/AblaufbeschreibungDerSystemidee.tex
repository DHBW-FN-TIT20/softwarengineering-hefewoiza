\section{Ablaufbeschreibung der Systemidee}
% periodisch vom Informationsportal
% Dabei Prüfung auf (konfigurierbarer Filter) Vegan (Tag) und Abgleich eigene Datenbank -> nur neue Items
% DATEN: Name, Restaurant-Name, Restaurant-Adresse, Rechnungsbezeichner (identisch zu Name auf Rechnung), Tags (vegan)
Für die Betreibung von \ac{aMRS} werden die folgenden Daten Periodisch (z.B.: jede fünf Minuten) vom Informationsportal abgerufen und in \ac{aMRS} persistent gespeichert:
Die zur eindeutigen Identifikation der Tagesgerichte benötigten Attribute Restaurantname, Restaurantadresse und Rechnungsbezeichner, sowie die Attribute Gerichtname und Klassifikationen (z.B.: Vegan).
Die bezogenen Daten werden in jeder Periode zunächst auf ausgewählte Klassifikationen (z.B.: Vegan) gefiltert und anschließend mit den bestehenden Daten in \ac{aMRS} verglichen, sodass das System ausschließlich um neue Datensätze erweitert wird.
Bei der Konfiguration von \ac{aMRS} können die zu filternden Klassifikationen konfiguriert werden, wodurch die durch den Auftraggeber geforderte Spezifizierung auf Vegane Gerichte ermöglicht wird.

% Übertragung der Daten / Bereitstellungsformat ist durch den Auftraggeber spezifizierbar, bzw. beliebig austauschbar
% -> Unser System kann die Daten über jedes Format entgegennehmen
Die Modularität von \ac{aMRS} ermöglicht einen Technologie unabhängigen Datenaustausch zwischen dem Informationsportal und \ac{aMRS}, wodurch die Bereitstellung der Daten durch das Informationsportal flexibel gestaltet ist.

% Ablauf Bewertungsprozess
Um den Bewertungsprozess auf der von \ac{aMRS} bereitgestellten Benutzeroberfläche zu starten, muss der Bewerter die Restaurantrechnung als Bild bereitstellen.
% TODO: evtl. umformulieren: Bewerter muss das Bild scannen (bereitstellen eventuell mit zu komplexem Ablauf assoziiert)
% TODO: Welche Form liefert das externe System zurück? -> Annahme: wir bekommen das fertige Objekt zurück, ergo wir haben einen Rechnungsscanner
Dieses wird anschließend an ein externes System übergeben, welches den Inhalt der Restaurantrechnung zurück liefert.
Die erhaltenen Daten sind dabei bereits in die folgenden Attribute untergliedert: Alle Rechnungsbezeichner die auf der Rechnung aufgelistet sind, den Restaurantnamen, die Restaurantadresse und die Rechnungsnummer.
Ausgehend von den erhaltenen Daten gleicht \ac{aMRS} die enthaltenen Gerichte mit den im System gespeicherten Daten ab.
Dabei werden die Gerichte, welche bereits mit der Rechnungsnummer bewertet wurden ebenfalls durch \ac{aMRS} identifiziert.
Somit kann \ac{aMRS} ermitteln, welche Gerichte für den Bewerter zur Bewertung freigegeben werden.
\ac{aMRS} leitet diese Information an die Benutzeroberfläche weiter, wodurch die Eingabe der Bewertung startet.

Durch das Absenden der Bewertung auf der Benutzeroberfläche wird die Bewertung zunächst durch \ac{aMRS} auf dessen Inhalt überprüft.
% TODO: Inhaltsprüfung aktuell eigenes System -> eventuell extern
Entspricht die erfasste Bewertung den festgelegten Kriterien, wird diese in \ac{aMRS} persistent gespeichert.


Über eine weitere Schnittstelle kann das Informationsportal die Bewertungen zu den Gerichten abrufen.
% TODO: Woraus besteht eine Bewertung? -> Nur Text? Text + Sterne? Sterne für verschiedene Kategorien?
Die durch \ac{aMRS} bereitgestellten Daten enthalten dabei die Attribute Restaurantname, Restaurantadresse, Rechnungsbezeichner und die Bewertung.
