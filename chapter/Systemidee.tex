\section{Systemidee}

Im Rahmen dieser \ac{OOA} soll das Softwaresystem \textit{\ac{aMRS}} konzeptioniert werden.
\newparagraph
Das Ziel des Systems ist die Verwirklichung der vom \hyperref[gls:auftraggeber]{Auftraggeber} gewünschten Erweiterung des Informationsportals zum Angebot von Tagesgerichten um eine \hyperref[gls:authentischeBewertung]{authentische Bewertung} veganer Gerichte.
Hierzu schlagen wir die \textit{\vFKW}, das nachfolgend spezifizierte Softwaresystem \textit{\ac{aMRS}} vor, welches den Bewertungsprozess von veganen Speisen realisiert und somit die gestellten Anforderungen abdeckt.
\textit{\ac{aMRS}} verwaltet vegane Speisen des Informationsportals und bietet Kunden die Möglichkeit, ihr Essen zu bewerten.
Diese Bewertungen können anschließend vom Informationsportal abgerufen werden.
Um sicherzustellen, dass die Bewertungen authentisch sind, verifiziert das System den Kunden als tatsächlichen Käufer und überprüft die Bewertungen zusätzlich auf ihre Inhalte.
\newparagraph
Das von uns konzipierte System \textit{\ac{aMRS}} erhält die veganen Tagesgerichte vom Informationsportal und speichert diese.
Somit wird die Bewertung der Gerichte zu einem späteren Zeitpunkt ermöglicht.
Erstellte Bewertungen können zusätzlich wiederkehrenden Tagesgerichten zugeordnet und bereitgestellt werden. Grundsätzlich werden sämtliche Daten persistent in \textit{\ac{aMRS}} gespeichert.
Dies macht \textit{\ac{aMRS}} erweiterbar und flexibel.
\newparagraph
Für den Bewertungsprozess bietet unsere Software-Lösung eine eigene Oberfläche, die parallel betrieben wird.
Somit müssen nur minimale (sonst teure) Anpassungen am bereits vorhandenen Informationsportal getätigt werden.
Die Käufer-Verifizierung erfolgt ebenfalls unabhängig vom Informationsportals des \hyperref[gls:auftraggeber]{Auftraggebers} über einen Scan der Restaurant-Rechnung.
Somit wird der Kauf verifiziert, ohne dass teure Anpassungen bei den Restaurants (z. B. ein Kassen-Plugin) nötig sind.

\subsection{Voraussetzungen an das Informationsportal}
Es wird vorausgesetzt, dass das System des \hyperref[gls:auftraggeber]{Auftraggebers} eine Schnittstelle bereitstellt, worüber die Daten zu den aktuellen Tagesgerichten abgerufen werden können.
Ein Gericht wird eindeutig durch einen Namen, ein Restaurantname, eine Restaurantadresse und einen \hyperref[gls:Rechnungsbezeichner]{Rechnungsbezeichner} identifiziert.
Der \hyperref[gls:Rechnungsbezeichner]{Rechnungsbezeichner} wird von den Restaurants definiert und ist identisch zu dem jeweiligen Bezeichner auf dem gedruckten Kassenbeleg.
Außerdem müssen die veganen Speisen als solche eindeutig gekennzeichnet sein.
\newparagraph
Um einen Bewertungsprozess zu starten, muss das Informationsportal eine Verknüpfung bzw. einen Verweis zu \textit{\ac{aMRS}} ermöglichen.
Teilnehmende Restaurants müssen Rechnungen nach der aktuellen deutschen Gesetzeslage ausstellen.

\subsection{Kosten}

% Teamgröße: 8 Personen
% Stundensatz: 180 €
% Tage: 168 Manntage
% Stunden pro Tag: 8 h
% Gesamt: 241.920 €

Für die Entwicklung der ersten Version von \textit{\ac{aMRS}} werden in unserem 8-köpfigen Team 21 Arbeitstage benötigt.
Der Stundensatz wird hierbei auf 180~€ festgelegt.
Bei einem Arbeitstag mit 8 Arbeitsstunden ergibt dies eine Gesamtsumme von 241.920~€ (inkl. Steuern).


