\section{Systemidee}

Im Rahmen dieser \ac{OOA} soll das Softwaresystem \ac{aMRS} konzeptioniert werden.
\newparagraph
Das Ziel des Systems ist die Verwirklichung der vom \hyperref[gls:auftraggeber]{Auftraggeber} gewünschten Erweiterung des \hyperref[gls:informationsportal]{Informationsportals} zum Angebot von \hyperref[gls:tagesgericht]{Tagesgerichten} um eine \hyperref[gls:authentischeBewertung]{authentische Bewertung} \hyperref[gls:vegan]{veganer} Gerichte.
Hierfür schlagen wir, die \vFKW, das nachfolgend spezifizierte Softwaresystem \ac{aMRS} vor, das den Bewertungsprozess von \hyperref[gls:vegan]{veganen} Speisen realisiert und somit die gestellten Anforderungen abdeckt.
\ac{aMRS} verwaltet \hyperref[gls:vegan]{vegane} Speisen des \hyperref[gls:informationsportal]{Informationsportals} und bietet Kunden die Möglichkeit, ihr Essen zu bewerten.
Diese Bewertungen können anschließend vom \hyperref[gls:informationsportal]{Informationsportal} abgerufen werden.
Um sicherzustellen, dass die Bewertungen authentisch sind, verifiziert das System den Kunden als tatsächlichen Käufer und überprüft die Bewertungen zusätzlich auf ihre Inhalte.
\newparagraph
Das von uns konzipierte System \ac{aMRS} erhält die \hyperref[gls:vegan]{veganen} \hyperref[gls:tagesgericht]{Tagesgerichte} vom \hyperref[gls:informationsportal]{Informationsportal} und speichert diese.
Somit wird die Bewertung der Gerichte zu einem späteren Zeitpunkt ermöglicht.
Erstellte Bewertungen können zusätzlich wiederkehrenden \hyperref[gls:tagesgericht]{Tagesgerichten} zugeordnet und bereitgestellt werden. Grundsätzlich werden sämtliche Daten persistent in \ac{aMRS} gespeichert.
Dies macht \ac{aMRS} erweiterbar und flexibel.
\newparagraph
Für den Bewertungsprozess bietet unsere Software-Lösung eine eigene Oberfläche, die parallel betrieben wird.
Somit müssen nur minimale (sonst teure) Anpassungen am bereits vorhandenen \hyperref[gls:informationsportal]{Informationsportal} getätigt werden.
Die Käufer-Verifizierung erfolgt ebenfalls unabhängig vom \hyperref[gls:informationsportal]{Informationsportals} des \hyperref[gls:auftraggeber]{Auftraggebers} über einen Scan der \hyperref[gls:restaurant]{Restaurant}-Rechnung.
Somit wird der Kauf verifiziert, ohne dass teure Anpassungen bei den \hyperref[gls:restaurant]{Restaurants} (z. B. ein Kassen-Plugin) nötig sind.

\subsection{Voraussetzungen an das Informationsportal}
Es wird vorausgesetzt, dass das System des \hyperref[gls:auftraggeber]{Auftraggebers} eine Schnittstelle bereitstellt, worüber die Daten zu den aktuellen \hyperref[gls:tagesgericht]{Tagesgerichten} abgerufen werden können.
Ein Gericht wird eindeutig durch einen Namen, ein \hyperref[gls:restaurantname]{Restaurantname}, eine \hyperref[gls:restaurantAdresse]{Restaurantadresse} und einen \hyperref[gls:Rechnungsbezeichner]{Rechnungsbezeichner} identifiziert.
Der \hyperref[gls:Rechnungsbezeichner]{Rechnungsbezeichner} wird von den \hyperref[gls:restaurant]{Restaurants} definiert und ist identisch zu dem jeweiligen Bezeichner auf dem gedruckten Kassenbeleg.
Außerdem müssen die \hyperref[gls:vegan]{veganen} Speisen als solche eindeutig gekennzeichnet sein.
\newparagraph
Um einen Bewertungsprozess zu starten, muss das \hyperref[gls:informationsportal]{Informationsportal} eine Verknüpfung bzw. einen Verweis zu \ac{aMRS} ermöglichen.
Teilnehmende \hyperref[gls:restaurant]{Restaurants} müssen Rechnungen nach der aktuellen deutschen Gesetzeslage ausstellen.

\subsection{Entwicklungsstufen}

\noindent Die erste Version von \ac{aMRS} deckt die oben beschriebenen Funktionen ab. \newline

\noindent Wenn gewünscht, kann das System in einer zweiten Entwicklungsstufe durch eine Benutzerverwaltung ergänzt werden.
Somit könnten Benutzer im Nachhinein ihre Bewertungen verändern oder löschen.

\subsection{Kosten}

% Teamgröße: 8 Personen
% Stundensatz: 180 €
% Tage: 168 Manntage
% Stunden pro Tag: 8 h
% Gesamt: 241.920 €

Für die Entwicklung der ersten Version von \ac{aMRS} werden in unserem 8-köpfigen Team 21 Arbeitstage benötigt.
Der Stundensatz wird hierbei auf 180~\euro{} festgelegt.
Bei einem Arbeitstag mit 8 Arbeitsstunden ergibt dies eine Gesamtsumme von 241.920~\euro{} (inkl. Steuern).


