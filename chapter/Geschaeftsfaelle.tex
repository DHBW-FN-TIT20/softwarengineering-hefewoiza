\section{Geschäftsfälle}\label{sec:Geschaeftsfaelle}

% - Vegane Tagesgerichte abrufen
% - Bewertungen von Tagesgerichten bereitstellen
% - Bewertungsprozess durchführen
% - Bewertender Benutzer als authentisch identifizieren
% - Rechnung analysieren
% - Bewertung auf Inhalt (Scam-/Spam) prüfen
% - Bewertung ablegen

\begin{table}[H]
    \centering
    \label{veganetagesgerichteabrufen}
    \begin{tabularx}{\textwidth}{| l | X |}
        \hline
        Name               & Vegane Tagesgerichte abrufen                                                                                                                  \\
        \hline
        Anwendungsfall-Typ & Geschäftsfall                                                                                                                                 \\
        \hline
        Kurzbeschreibung   & Alle bewertbaren Tagesgerichte werden vom Informationsportal abgerufen und verarbeitet. Nur diese sind anschließend für den Kunden bewertbar. \\
        \hline
        Auslöser           & Periodisch                                                                                                                                    \\
        \hline
        Ergebnis           & Alle benötigten Daten für die Bewertung der Tagesgerichte sind im aMRS gespeichert.                                                           \\
        \hline
        Akteure            & Informationsportal, \ac{aMRS}                                                                                                                 \\
        \hline
    \end{tabularx}
\end{table}


\begin{table}[H]
    \centering
    \label{bewertungenbereitstellen}
    \begin{tabularx}{\textwidth}{| l | X |}
        \hline
        Name               & Bewertungen von Tagesgerichten bereitstellen                                                                                                  \\
        \hline
        Anwendungsfall-Typ & Geschäftsfall                                                                                                                                 \\
        \hline
        Kurzbeschreibung   & Bewertungen können von dem Infoportal abgerufen werden. Diese werden entsprechend ausgelesen und bereitgestellt. \\
        \hline
        Auslöser           & Das Informationsportal möchte eine Bewertung zu einem Tagesgericht                                                                                                                             \\
        \hline
        Ergebnis           & Eine oder mehrere Bewertungen werden zu dem Infoportal zurückgeschickt.                                                                       \\
        \hline
        Akteure            & Infoportal, \ac{aMRS}                                                                                                                         \\
        \hline
    \end{tabularx}
\end{table}

\begin{table}[H]
    \centering
    \label{bewertungsprozessDurchfuehren}
    \begin{tabularx}{\textwidth}{| l | X |}
        \hline
        Name               & Bewertungsprozess durchführen                                                                                                                                                                                                                                                                                                                                                                                                                       \\
        \hline
        Anwendungsfall-Typ & Geschäftsfall                                                                                                                                                                                                                                                                                                                                                                                                                                       \\
        \hline
        Kurzbeschreibung   & Ein Bewertungsprozess zur Bewertung von Tagesgerichten wird gestartet. Zur Sicherstellung der Authentizität der Bewertung müssen die Benutzer die Restaurantrechnung als Bild bereitstellen. Dieses wird anschließend von einem externen OCR-System analysiert. Die erhaltenen Daten werden mit den gespeicherten Daten abgeglichen. War diese Überprüfung erfolgreich, bekommt der Benutzer die Möglichkeit Bewertungen für die Tagesgerichte, welche auf der Rechnung aufgelistet sind, zu erstellen. Der Inhalt der Bewertung wird anschließend geprüft. Abschließend wird die Bewertung persistent gespeichert. \\
        \hline
        Auslöser           & Benutzer will eine Bewertung verfassen                                                                                                                                                                                                                                                                                                                                                                                                              \\
        \hline
        Ergebnis           & Eine authentische Bewertung eines veganen Tagesgerichts wurde verfasst                                                                                                                                                                                                                                                                                                                                                                              \\
        \hline
        Akteure            & Benutzer, \ac{aMRS}                                                                                                                                                                                                                                                                                                                                                                                                                                 \\
        \hline
    \end{tabularx}
\end{table}



\begin{table}[H]
    \centering
    \label{}
    \begin{tabularx}{\textwidth}{| l | X |}
        \hline
        Name               & Bewertenden Benutzer als authentisch identifizieren                                                                                                                                                                              \\
        \hline
        Anwendungsfall-Typ & Geschäftsfall                                                                                                                                                                                                                    \\
        \hline
        Kurzbeschreibung   & Der Benutzer wird dazu aufgefordert ein Foto seiner Restaurantrechnung zu machen und dieses hochzuladen. Anschließend startet die Analyse der Rechnung. Die Analysedaten werden zur Überprüfung der Authentizität herangezogen. \\
        \hline
        Auslöser           & Bewertungsprozess gestartet                                                                                                                                                                                                      \\
        \hline
        Ergebnis           & Benutzer ist als authentisch identifiziert                                                                                                                                                                                       \\
        \hline
        Akteure            & Benutzer, \ac{aMRS}                                                                                                                                                                                                              \\
        \hline
    \end{tabularx}
\end{table}

\begin{table}[H]
    \centering
    \label{}
    \begin{tabularx}{\textwidth}{| l | X |}
        \hline
        Name               & Restaurantrechnung analysieren                                                                                                                                                       \\
        \hline
        Anwendungsfall-Typ & Geschäftsfall                                                                                                                                                                         \\
        \hline
        Kurzbeschreibung   & Die Restaurantrechnung wird auf das Vorhandensein aller Pflichtfelder überprüft. Die Analysedaten werden in ein Format gebracht, welches zur Weiterverarbeitung genutzt werden kann. \\
        \hline
        Auslöser           & Restaurantrechnung hochgeladen                                                                                                                                                       \\
        \hline
        Ergebnis           & Restaurantrechnung ist valide und Daten stehen zur Verfügung                                                                                                                         \\
        \hline
        Akteure            & \ac{aMRS}, OCR-System                                                                                                                                                                 \\
        \hline
    \end{tabularx}
\end{table}


\begin{table}[H]
    \centering
    \label{bewertungPruefen}
    \begin{tabularx}{\textwidth}{| l | X |}
        \hline
        Name               & Bewertung auf Inhalt prüfen                                                                                                   \\
        \hline
        Anwendungsfall-Typ & Geschäftsfall                                                                                                                 \\
        \hline
        Kurzbeschreibung   & Bevor eine Bewertung in \ac{aMRS} gesichert wird, wird die Bewertung auf zum Beispiel Spam, Scam und politisch korrekten Inhalt überprüft. \\
        \hline
        Auslöser           & Erhalten einer neuen Bewertung                                                                                                \\
        \hline
        Ergebnis           & Die Bewertung darf gesichert und angezeigt werden.                                                                            \\
        \hline
        Akteure            & \ac{aMRS}                                                                                                                     \\
        \hline
    \end{tabularx}
\end{table}


\begin{table}[H]
    \centering
    \label{bewertungablegen}
    \begin{tabularx}{\textwidth}{| l | X |}
        \hline
        Name               & Bewertung ablegen                                                                   \\
        \hline
        Anwendungsfall-Typ & Geschäftsfall                                                                       \\
        \hline
        Kurzbeschreibung   & Falls eine valide Bewertung eintrifft, wird diese persistent abgespeichert. \\
        \hline
        Auslöser           & Eintreffende Bewertung                                                              \\
        \hline
        Ergebnis           & Die Bewertung wird persistent gespeichert                                        \\
        \hline
        Akteure            & \ac{aMRS}                                                       \\
        \hline
    \end{tabularx}
\end{table}

