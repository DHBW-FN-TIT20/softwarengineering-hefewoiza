\section{Geschäftsfälle}\label{sec:Geschaeftsfaelle}

% - Vegane Tagesgerichte abrufen
% - Bewertungen von Tagesgerichten bereitstellen
% - Bewertungsprozess durchführen
% - Bewertender Benutzer als authentisch identifizieren
% - Rechnung analysieren
% - Bewertung auf Inhalt (Scam-/Spam) prüfen
% - Bewertung ablegen

\begin{table}[H]
    \centering
    \label{veganetagesgerichteabrufen}
    \begin{tabularx}{\textwidth}{| l | X |}
        \hline
        Name               & Vegane \hyperref[gls:tagesgericht]{Tagesgerichte} abrufen                                                                                                                                                     \\
        \hline
        Anwendungsfall-Typ & Geschäftsfall                                                                                                                                                                                                 \\
        \hline
        Kurzbeschreibung   & Alle bewertbaren \hyperref[gls:tagesgericht]{Tagesgerichte} werden vom \hyperref[gls:informationsportal]{Informationsportal} abgerufen und verarbeitet. Nur diese sind anschließend für den Kunden bewertbar. \\
        \hline
        Auslöser           & Periodisch                                                                                                                                                                                                    \\
        \hline
        Ergebnis           & Alle benötigten Daten für die Bewertung der \hyperref[gls:tagesgericht]{Tagesgerichte} sind im aMRS gespeichert.                                                                                              \\
        \hline
        Akteure            & \hyperref[gls:informationsportal]{Informationsportal}, \ac{aMRS}                                                                                                                                              \\
        \hline
    \end{tabularx}
\end{table}


\begin{table}[H]
    \centering
    \label{bewertungenbereitstellen}
    \begin{tabularx}{\textwidth}{| l | X |}
        \hline
        Name               & Bewertungen von \hyperref[gls:tagesgericht]{Tagesgerichten} bereitstellen                                                                                                                                      \\
        \hline
        Anwendungsfall-Typ & Geschäftsfall                                                                                                                                                                                                  \\
        \hline
        Kurzbeschreibung   & Bewertungen zu einem \hyperref[gls:tagesgericht]{Tagesgericht} können von dem \hyperref[gls:informationsportal]{Informationsportal} abgerufen werden. Diese werden entsprechend ausgelesen und bereitgestellt. \\
        \hline
        Auslöser           & Das \hyperref[gls:informationsportal]{Informationsportal} möchte alle Bewertung zu einem \hyperref[gls:tagesgericht]{Tagesgericht}                                                                             \\
        \hline
        Ergebnis           & Eine oder mehrere Bewertungen werden zu dem \hyperref[gls:informationsportal]{Informationsportal} zurückgeschickt.                                                                                             \\
        \hline
        Akteure            & \hyperref[gls:informationsportal]{Informationsportal}, \ac{aMRS}                                                                                                                                               \\
        \hline
    \end{tabularx}
\end{table}

\begin{table}[H]
    \centering
    \label{bewertungsprozessDurchfuehren}
    \begin{tabularx}{\textwidth}{| l | X |}
        \hline
        Name               & Bewertungsprozess durchführen                                                                                                                                                                                                                                                                                                                                                                                                                                                                                                                                                                                                                                                                                                     \\
        \hline
        Anwendungsfall-Typ & Geschäftsfall                                                                                                                                                                                                                                                                                                                                                                                                                                                                                                                                                                                                                                                                                                                     \\
        \hline
        Kurzbeschreibung   & Ein Bewertungsprozess zur Bewertung von \hyperref[gls:tagesgericht]{Tagesgerichten} wird gestartet. Zur Sicherstellung der Authentizität der Bewertung müssen die \hyperref[gls:nutzer]{Benutzer} die \hyperref[gls:restaurantRechnung]{Restaurantrechnung} als Bild bereitstellen. Dieses wird anschließend von einem externen OCR-System analysiert. Die erhaltenen Daten werden mit den gespeicherten Daten abgeglichen. War diese Überprüfung erfolgreich, bekommt der \hyperref[gls:nutzer]{Benutzer} die Möglichkeit Bewertungen für die Tagesgerichte, welche auf der Rechnung aufgelistet sind, zu erstellen. Der Inhalt der Bewertung wird anschließend geprüft. Abschließend wird die Bewertung persistent gespeichert. \\
        \hline
        Auslöser           & \hyperref[gls:nutzer]{Benutzer} will eine Bewertung verfassen                                                                                                                                                                                                                                                                                                                                                                                                                                                                                                                                                                                                                                                                     \\
        \hline
        Ergebnis           & Eine \hyperref[gls:authentischeBewertung]{authentische Bewertung} eines veganen \hyperref[gls:tagesgericht]{Tagesgerichts} wurde verfasst                                                                                                                                                                                                                                                                                                                                                                                                                                                                                                                                                                                         \\
        \hline
        Akteure            & \hyperref[gls:nutzer]{Benutzer}, \ac{aMRS}                                                                                                                                                                                                                                                                                                                                                                                                                                                                                                                                                                                                                                                                                        \\
        \hline
    \end{tabularx}
\end{table}



\begin{table}[H]
    \centering
    \label{}
    \begin{tabularx}{\textwidth}{| l | X |}
        \hline
        Name               & Bewertenden \hyperref[gls:nutzer]{Benutzer} als authentisch identifizieren                                                                                                                                                                                                                \\
        \hline
        Anwendungsfall-Typ & Geschäftsfall                                                                                                                                                                                                                                                                             \\
        \hline
        Kurzbeschreibung   & Der \hyperref[gls:nutzer]{Benutzer} wird dazu aufgefordert ein Foto seiner \hyperref[gls:restaurantRechnung]{Restaurantrechnung} zu machen und dieses hochzuladen. Anschließend startet die Analyse der Rechnung. Die Analysedaten werden zur Überprüfung der Authentizität herangezogen. \\
        \hline
        Auslöser           & Bewertungsprozess gestartet                                                                                                                                                                                                                                                               \\
        \hline
        Ergebnis           & \hyperref[gls:nutzer]{Benutzer} ist als authentisch identifiziert                                                                                                                                                                                                                         \\
        \hline
        Akteure            & \hyperref[gls:nutzer]{Benutzer}, \ac{aMRS}                                                                                                                                                                                                                                                \\
        \hline
    \end{tabularx}
\end{table}

\begin{table}[H]
    \centering
    \label{}
    \begin{tabularx}{\textwidth}{| l | X |}
        \hline
        Name               & \hyperref[gls:restaurantRechnung]{Restaurantrechnung} analysieren                                                                                                                                                       \\
        \hline
        Anwendungsfall-Typ & Geschäftsfall                                                                                                                                                                                                           \\
        \hline
        Kurzbeschreibung   & Die \hyperref[gls:restaurantRechnung]{Restaurantrechnung} wird auf das Vorhandensein aller Pflichtfelder überprüft. Die Analysedaten werden in ein Format gebracht, welches zur Weiterverarbeitung genutzt werden kann. \\
        \hline
        Auslöser           & \hyperref[gls:restaurantRechnung]{Restaurantrechnung} hochgeladen                                                                                                                                                       \\
        \hline
        Ergebnis           & \hyperref[gls:restaurantRechnung]{Restaurantrechnung} ist valide und Daten stehen zur Verfügung                                                                                                                         \\
        \hline
        Akteure            & \ac{aMRS}, OCR-System                                                                                                                                                                                                   \\
        \hline
    \end{tabularx}
\end{table}


\begin{table}[H]
    \centering
    \label{bewertungPruefen}
    \begin{tabularx}{\textwidth}{| l | X |}
        \hline
        Name               & Bewertung auf Inhalt prüfen                                                                                                                \\
        \hline
        Anwendungsfall-Typ & Geschäftsfall                                                                                                                              \\
        \hline
        Kurzbeschreibung   & Bevor eine Bewertung in \ac{aMRS} gesichert wird, wird die Bewertung auf zum Beispiel Spam, Scam und politisch korrekten Inhalt überprüft. \\
        \hline
        Auslöser           & Erhalten einer neuen Bewertung                                                                                                             \\
        \hline
        Ergebnis           & Die Bewertung darf gesichert und angezeigt werden.                                                                                         \\
        \hline
        Akteure            & \ac{aMRS}                                                                                                                                  \\
        \hline
    \end{tabularx}
\end{table}


\begin{table}[H]
    \centering
    \label{bewertungablegen}
    \begin{tabularx}{\textwidth}{| l | X |}
        \hline
        Name               & Bewertung ablegen                                                           \\
        \hline
        Anwendungsfall-Typ & Geschäftsfall                                                               \\
        \hline
        Kurzbeschreibung   & Falls eine valide Bewertung eintrifft, wird diese persistent abgespeichert. \\
        \hline
        Auslöser           & Eintreffende Bewertung                                                      \\
        \hline
        Ergebnis           & Die Bewertung wird persistent gespeichert                                   \\
        \hline
        Akteure            & \ac{aMRS}                                                                   \\
        \hline
    \end{tabularx}
\end{table}

