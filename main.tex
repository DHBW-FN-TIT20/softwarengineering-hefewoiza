%%%%%%%%%%%%%%%%%%%%%%%%%%%%%%%%%%%%%%%%%%%%%%%%%%%%%%%%%%%%%%%%%%%%
%%%           Vorlage für eine Ausarbeitung an der DHBW          %%%
%%%                                                              %%%
%%%      Bereiche die bearbeitet werden müssen werden durch      %%%
%%%      einen solchen Kommentarblock eingeleitet und enden      %%%
%%%      mit der nächsten Trennlinie.                            %%%
%%%                                                              %%%
%%%      In dieser Datei müssen folgende Bereiche bearbeitet     %%%
%%%      werden:                                                 %%%
%%%      - Angaben zur Arbeit                                    %%%
%%%      - EIGENE KAPITEL EINFÜGEN                               %%%
%%%                                                              %%%
%%%      Benötigte Seiten und Verzeichnisse können unter         %%%
%%%      "Einführung und Verzeichnisse" ein- bzw. auskommentiert %%%
%%%      werden.                                                 %%%
%%%                                                              %%%
%%%%%%%%%%%%%%%%%%%%%%%%%%%%%%%%%%%%%%%%%%%%%%%%%%%%%%%%%%%%%%%%%%%%

\documentclass[a4paper,12pt]{article}
\usepackage[left=2.5cm,right=2.5cm,top=2.5cm,bottom=2.5cm,includehead]{geometry}      % Einstellungen der Seitenränder
\usepackage[english, ngerman]{babel}                                                  % deutsche Silbentrennung
\usepackage[utf8]{inputenc}                                                           % Umlaute
\usepackage[T1]{fontenc}													                                    % Umlaute auch richtig ausgeben
\usepackage{newtxtext,newtxmath}                                                      % Font = Times New Roman
\usepackage{hyperref}
\usepackage[nottoc]{tocbibind}
\usepackage{fancyhdr}
\usepackage{setspace}
\usepackage[backend=bibtex, citestyle=authoryear, bibstyle=authoryear]{biblatex}      % Bibliothek für Zitate
\usepackage{csquotes}                                                                 % Zusatzpacket für Zitate
\usepackage{amsmath}                                                                  % Zurücksetzen der Tabellen- und Abbildungsnummerierung je Sektion
\usepackage[labelfont=bf,aboveskip=1mm]{caption}                                      % Bild- und Tabellenunterschrift (fett)
\usepackage[bottom,multiple,hang,marginal]{footmisc}                                  % Fußnoten [Ausrichtung unten, Trennung durch Seperator bei mehreren Fußnoten]
\usepackage{graphicx}  
\graphicspath{{./images/}}                                                            % Grafiken
\usepackage[dvipsnames]{xcolor}                                                       % Farbige Buchstaben
\usepackage{wrapfig}                                                                  % Bilder in Text integrieren
\usepackage{enumitem}                                                                 % Befehl setlist (Zeilenabstand für itemize Umgebung auf 1 setzen)
\usepackage{listings}                                                                 % Quelltexte
\definecolor{commentgreen}{RGB}{87,166,74}                                            % Kommentar-Farbe für Quellcode
\lstset{numbers=left, numberstyle=\tiny, numbersep=8pt, frame=single, framexleftmargin=15pt, breaklines=true, commentstyle=\color{commentgreen}}
\usepackage{tabularx}                                                                 % Tabellen
\usepackage{multirow}                                                                 % Mehrzeilige Tabelleneinträge
\usepackage[addtotoc]{abstract}                                                       % Abstract
\usepackage[nohyperlinks, printonlyused, withpage]{acronym}                           % Abkürzungen
\usepackage{dirtree}                                                                  % Ordnerstruktur (z.B. für Anhang)
\usepackage{float}
\usepackage{pdfpages}

%%%%%%%%%%%%%%%%%%%%%%%%%%%%%%%%%%%%%%%%%%%%%%%%%%%%%%%%%%%%%%%%%%%%
%%%                      Angaben zur Arbeit                      %%%
%%%%%%%%%%%%%%%%%%%%%%%%%%%%%%%%%%%%%%%%%%%%%%%%%%%%%%%%%%%%%%%%%%%%
\def\vFirmenlogoPfad{}                  %% relativer Pfad Bsp.: images/Firmenlogo.png
\def\vDHBWLogoPfad{images/DHBW_logo.jpg}                          %% relativer Pfad Bsp.: images/DHBW_logo.jpg
\def\vUnterschrift{}                    %% Pfad zu Bild mit Unterschrift (für digitale Abgabe) Bsp.: images/Unterschrift.png

\def\vTitel{Software Engineering 2}                           %% 
\def\vUntertitel{}                      %% 
\def\vArbeitstyp{Hausarbeit}                      %% Projektarbeit/Seminararbeit/Bachelorarbeit
\def\vArbeitsbezeichnung{}              %% T1000/T2000/T3000

\def\vLB{Lukas Braun}
\def\vJB{Johannes Brandenburger}
\def\vDF{David Felder}
\def\vFG{Florian Glaser}
\def\vFH{Florian Herkommer}
\def\vPP{Phillipp Patzelt}
\def\vHS{Henry Schuler}
\def\vBS{Baldur Siegel}



\def\vAutor{\vJB, \vLB, \vDF, \vFG, \vFH, \vPP, \vHS, \vBS}                           %% Vorname Nachname
\def\vMatrikelnummer{}                  %% 7-stellige Zahl
\def\vKursKuerzel{TIT20}                     %% Bsp.: TIT20
\def\vPhasenbezeichnung{Theoriephasen}               %% Praxisphase/Theoriephase
\def\vStudienJahr{dritte}                     %% erste/zweite/dritte
\def\vDHBWStandort{Ravensburg}                    %% Bsp.: Ravensburg
\def\vDHBWCampus{Friedrichshafen}                      %% Bsp.: Friedrichshafen
\def\vFakultaet{Technik}                       %% Technik/Wirtschaft
\def\vStudiengang{Informatik}                     %% Informationstechnik/...
\def\vKurs{TIT20}                     %% IT/...

\def\vBearbeitungsort{Friedrichshafen}                 %%                       %% 
\def\vBetreuer{Prof. Dr. Andreas Judt}                        %% Vorname Nachname

\def\vAbgabedatum{\today}               %% DD. MONTH YYYY
\def\vBearbeitungszeitraum{01.10.2022 - 23.12.2022}            %% DD.MM.YYYY - DD.MM.YYYY
%TODO Datum anpassen

%%%%%%%%%%%%%%%%%%%%%%%%% Eigene Kommandos %%%%%%%%%%%%%%%%%%%%%%%%%
% Definition von \gqq{}: Text in Anführungszeichen
\newcommand{\gqq}[1]{\glqq #1\grqq}
% Definition von \gq{}: Text in Anführungszeichen
\newcommand{\gq}[1]{\glq #1\grq}
% Spezielle Hervorhebung von Schlüsselwörtern
\newcommand{\textOrdner}[1]{\texttt{#1}}
\newcommand{\textVariable}[1]{\texttt{#1}}
\newcommand{\textKlasse}[1]{\texttt{#1}}
\newcommand{\textFunktion}[1]{\texttt{#1}}
\newcommand{\newparagraph}{\newline \newline}
% Quellenangabe bei Bildern
\newcommand{\customcaption}[2]{\caption[#1]{ #1. #2.}}
\def\vFKW{FKW Software Solutions }

%%%%%%%%%%%%%%%%%%%% Zitatbibliothek einbinden %%%%%%%%%%%%%%%%%%%%%
\addbibresource{./literatur/literatur.bib}


%%%%%%%%%%%%%%%%%%%%%%%% PDF-Einstellungen %%%%%%%%%%%%%%%%%%%%%%%%%
\hypersetup{
  bookmarksopen=false,
	bookmarksnumbered=true,
	bookmarksopenlevel=0,
  pdftitle=\vTitel,
  pdfsubject=\vTitel,
  pdfauthor=\vAutor,
  pdfborder={0 0 0},
	pdfstartview=Fit,
  pdfpagelayout=SinglePage
}


%%%%%%%%%%%%%%%%%%%%%%%% Kopf- und Fußzeile %%%%%%%%%%%%%%%%%%%%%%%%
\pagestyle{fancy}
\setlength{\headheight}{15pt}
\fancyhf{}
\fancyhead[R]{\thepage}


%%%%%%%%%%%%%%%%%%%%%%%%%%%%%% Layout %%%%%%%%%%%%%%%%%%%%%%%%%%%%%%
\onehalfspacing
\setlist{noitemsep}

\addto\captionsngerman{
  \renewcommand{\figurename}{Abb.}
  \renewcommand{\tablename}{Tab.}
}
\numberwithin{table}{section}                               % Tabellennummerierung je Sektion zurücksetzen
\numberwithin{figure}{section}                              % Abbildungsnummerierung je Sektion zurücksetzen
\renewcommand{\thetable}{\arabic{section}.\arabic{table}}   % Tabellennummerierung mit Section
\renewcommand{\thefigure}{\arabic{section}.\arabic{figure}} % Abbildungsnummerierung mit Section
\renewcommand{\thefootnote}{\arabic{footnote}}              % Sektionsbezeichnung von Fußnoten entfernen

\renewcommand{\multfootsep}{, }                             % Mehrere Fußnoten durch ", " trennen


%%%%%%%%%%%%%%%%%%%%%%%%%%%%% Dokument %%%%%%%%%%%%%%%%%%%%%%%%%%%%%

\begin{document}


  %%%%%%%%%%%%%%%%%%% Einführung und Verzeichnisse %%%%%%%%%%%%%%%%%%%
  \pagenumbering{Roman}

  \begin{titlepage}
  \begin{minipage}{6in}
    \vspace*{-2cm}
    \centering
    \hspace{-2cm}
	\ifx\vFirmenlogoPfad\empty
	\else
    \raisebox{-0.5\height}{\includegraphics[height=4cm]{\vFirmenlogoPfad}}
  \fi
	\hfill
	\ifx\vDHBWLogoPfad\empty
	\else
   	\raisebox{-0.5\height}{\includegraphics[height=4cm]{\vDHBWLogoPfad}}
	\fi
  \end{minipage}
  \begin{center}
    \vspace*{0.5cm}
    \Huge\textbf{\vTitel}\\
		\ifx\vUntertitel\empty
		\else
			\Large\rm\vUntertitel\\
		\fi
		\vspace*{2cm}
		\Large\textbf{\vArbeitstyp}
		\ifx\vArbeitsbezeichnung\empty
		\else
			\textbf{\vArbeitsbezeichnung}
		\fi
		\\
		\normalsize
		über die \vPhasenbezeichnung\ des \vStudienJahr{n}\ Studienjahrs \\
		\vspace*{1cm}
		an der Fakultät für \vFakultaet\\
		im Studiengang \vStudiengang\\
		\vspace*{0.5cm}
		an der DHBW \vDHBWStandort\\
		\ifx\vDHBWCampus\empty
		\else
		Campus \vDHBWCampus\\
		\fi
		\vspace*{0.5cm}
		von\\
		\ifx\vAutor\empty
		\else
			\vAutor\\
		\fi
		\vspace*{1cm}
		\vAbgabedatum
		\vfill
  \end{center}
  \begin{tabular}{ll}
    Bearbeitungszeitraum:&\vBearbeitungszeitraum\\
    Kurs:&\vKurs\\
	  Dozent der Hochschule:&\vBetreuer\\
  \end{tabular}
\end{titlepage}
\newpage
\setcounter{page}{2}
  % \thispagestyle{empty}
\section*{\Huge{Sperrvermerk}}

\addcontentsline{toc}{section}{Sperrvermerk}
gemäß Ziffer 1.1.13 der Anlage 1 zu §§ 3, 4 und 5  der Studien- und Prüfungsordnung für die Bachelorstudiengänge im Studienbereich Technik der Dualen Hochschule Baden-Würt­tem­berg vom 29.09.2017.\\

\noindent \gqq{Der Inhalt dieser Arbeit darf weder als Ganzes noch in Auszügen Personen außerhalb des Prüfungsprozesses und des Evaluationsverfahrens zugänglich gemacht werden, sofern keine anders lautende Genehmigung vom Dualen Partner vorliegt.}

\vfill
\leavevmode
\newline
\parbox{6cm}{\strut\centering \vBearbeitungsort, \vAbgabedatum\hrule\strut\centering\footnotesize Ort, Datum} 
\hfill
\ifx\vUnterschrift\empty
\parbox{6cm}{\strut\hspace{1pt} \vAbteilung\hrule\strut\centering\footnotesize Abteilung, Unterschrift}
\else
\parbox{6cm}{\strut\hspace{1pt} \vAbteilung, \parbox[b]{3cm}{\vspace{-10cm}\includegraphics[width=3cm]{\vUnterschrift}}\hrule\strut\centering\footnotesize Abteilung, Unterschrift}
\fi
\vspace{1cm}

\newpage
  \thispagestyle{empty}
\section*{\Huge{Gender Erklärung}}

\addcontentsline{toc}{section}{Gendererklärung}
Aus Gründen der besseren Lesbarkeit wird in dieser Bachelorarbeit auf die gleichzeitige Verwendung der Sprachformen männlich,
weiblich und divers (m/w/d) verzichtet. Sämtliche Formulierungen gelten gleichermaßen für alle Geschlechter.
\newpage
  \thispagestyle{empty}
\section*{\Huge{Selbstständigkeitserklärung}}

\addcontentsline{toc}{section}{Selbstständigkeitserklärung}
gemäß Ziffer 1.1.13 der Anlage 1 zu §§ 3, 4 und 5  der Studien- und Prüfungsordnung für die Bachelorstudiengänge im Studienbereich Technik der Dualen Hochschule Baden-Würt­tem­berg vom 29.09.2017.

\noindent Wir versichern hiermit, dass wir unsere Bachelorarbeit (bzw. Projektarbeit oder Studienarbeit bzw. Hausarbeit) mit dem Thema: 
\begin{center}
	\Large\textbf{\vTitel}
\end{center}
selbstständig verfasst und keine anderen als die angegebenen Quellen und Hilfsmittel benutzt haben. Wir versichern zudem, dass die eingereichte elektronische Fassung mit der gedruckten Fassung übereinstimmt.

\vfill
\leavevmode
\newline
\parbox{7cm}{\strut\centering \vBearbeitungsort, \vAbgabedatum\hrule\strut\centering\footnotesize Ort, Datum} 
\hfill
\parbox{7cm}{\strut\hspace{1pt} \hrule\strut\centering\footnotesize \vJB}
\newline
\vspace{1cm}
\newline
\parbox{7cm}{\strut\centering \vBearbeitungsort, \vAbgabedatum\hrule\strut\centering\footnotesize Ort, Datum} 
\hfill
\parbox{7cm}{\strut\hspace{1pt} \hrule\strut\centering\footnotesize \vLB}
\newline
\vspace{1cm}
\newline
\parbox{7cm}{\strut\centering \vBearbeitungsort, \vAbgabedatum\hrule\strut\centering\footnotesize Ort, Datum} 
\hfill
\parbox{7cm}{\strut\hspace{1pt} \hrule\strut\centering\footnotesize \vDF}
\newline
\vspace{1cm}
\newline
\parbox{7cm}{\strut\centering \vBearbeitungsort, \vAbgabedatum\hrule\strut\centering\footnotesize Ort, Datum} 
\hfill
\parbox{7cm}{\strut\hspace{1pt} \hrule\strut\centering\footnotesize \vFG}
\newline
\vspace{1cm}
\newline
\parbox{7cm}{\strut\centering \vBearbeitungsort, \vAbgabedatum\hrule\strut\centering\footnotesize Ort, Datum} 
\hfill
\parbox{7cm}{\strut\hspace{1pt} \hrule\strut\centering\footnotesize \vFH}
\newline
\vspace{1cm}
\newline
\parbox{7cm}{\strut\centering \vBearbeitungsort, \vAbgabedatum\hrule\strut\centering\footnotesize Ort, Datum} 
\hfill
\parbox{7cm}{\strut\hspace{1pt} \hrule\strut\centering\footnotesize \vPP}
\newline
\vspace{1cm}
\newline
\parbox{7cm}{\strut\centering \vBearbeitungsort, \vAbgabedatum\hrule\strut\centering\footnotesize Ort, Datum} 
\hfill
\parbox{7cm}{\strut\hspace{1pt} \hrule\strut\centering\footnotesize \vHS}
\newline
\vspace{1cm}
\newline
\parbox{7cm}{\strut\centering \vBearbeitungsort, \vAbgabedatum\hrule\strut\centering\footnotesize Ort, Datum} 
\hfill
\parbox{7cm}{\strut\hspace{1pt} \hrule\strut\centering\footnotesize \vBS}
\newpage
  %\phantomsection
\newenvironment{keywords}{
	\begin{flushleft}
	\small	
	\textbf{
		\iflanguage{ngerman}{Schlüsselwörter}{\iflanguage{english}{Keywords}{}}
	}
}{\end{flushleft}}

% Deutsche Zusammenfassung
\begin{abstract}
	
\end{abstract}

% Schlüsselwörter Deutsch
\begin{keywords}
	
\end{keywords}


\selectlanguage{english}
% Englisches Abstract
\begin{abstract}

\end{abstract}

% Schlüsselwörter Englisch
\begin{keywords}

\end{keywords}


\selectlanguage{ngerman}
\newpage
  \pdfbookmark[1]{\contentsname}{toc}
\tableofcontents
\newpage
  \section*{Abkürzungsverzeichnis}
\addcontentsline{toc}{section}{Abkürzungsverzeichnis}
\begin{acronym}
  \acro{DHBW}[DHBW]{Duale Hochschule Ba\-den-\-Würt\-tem\-berg}
  \acroplural{DHBW}[DHBW]{Dualen Hochschule Ba\-den-\-Würt\-tem\-berg}
  \acro{AWS}[AWS]{Amazon Web Services}
\end{acronym}
\newpage
  \listoffigures
\newpage
  \listoftables
\newpage
  \lstlistoflistings
\addcontentsline{toc}{section}{Listings}
\newpage
  % \section*{Vorwort}
\addcontentsline{toc}{section}{Vorwort}
\newpage


  %%%%%%%%%%%%%%%%%%%%%%%%%%%%% Kapitel %%%%%%%%%%%%%%%%%%%%%%%%%%%%%%
  \pagestyle{fancy}
  \fancyhead[L]{\nouppercase{\rightmark}}    % Abschnittsname im Header
  \pagenumbering{arabic}

  %%%%%%%%%%%%%%%%%%%%%%%%%%%%%%%%%%%%%%%%%%%%%%%%%%%%%%%%%%%%%%%%%%%%
  %%%%                   EIGENE KAPITEL EINFÜGEN                  %%%%
  %%%%%%%%%%%%%%%%%%%%%%%%%%%%%%%%%%%%%%%%%%%%%%%%%%%%%%%%%%%%%%%%%%%%
  \section{Einleitung}

  \section{Technische Machbarkeit}
Die technische Machbarkeit ist ein wichtiger Aspekt bei der Entwicklung eines Software-Systems, da sie darüber entscheidet, ob das System überhaupt entwickelt werden kann und welche Technologien dafür verwendet werden müssen. Daher ist es wichtig, die technische Machbarkeit frühzeitig im Entwicklungsprozess zu untersuchen, um sicherzustellen, dass das System erfolgreich entwickelt werden kann.
Die Analyse der technischen Machbarkeit erfolgt unter der Beachtung der fünf wesentlichen Einflussfaktoren für die Auswahl von Technologien: Entwickler/Zulieferer, Unternehmen, Markt, Zeit und Gesetze.
\newparagraph
Aus diesen Einflussfaktoren werden im folgenden Kapitel spezielle Ausschlusskriterien für Technologien abgeleitet.
Abnschließend wird eine Technologie anhand der Kriterien und den in den vorherigen Kapiteln konzeptionierten Komponenten und Abläufen empfohlen, mit der die Anforderungen des Auftraggebers umgesetzt werden können.

\subsection{Definition der Ausschlusskriterien}
Es ist wichtig, dass die Technologien, die für das System verwendet werden, technologisch unabhängig sind. Das bedeutet, dass sie nicht von anderen Systemen abhängig sind und problemlos in jedes System integriert werden können. Die Unabhängigkeit wird vor allem deshalb vorausgesetzt, da das System des Auftraggebers völlig unbekannt ist und sich das neue System problemlos in die vorhandene Architektur einfügen können muss.
\newparagraph
Ein weiteres wichtiges Kriterium ist das Vorwissen der Entwickler. In diesem Fall verfügen die Entwickler der \vFKW bereits über umfangreiche Kenntnisse in den Webtechnologien React, TypeScript, JavaScript, Next.js und Node.js. Auch Kenntnisse in den Backend-Technologien C\#, Python, MariaDB, MySQL und PostgreSQL sowie in der Cloud-Technologie \ac{AWS} sind bereits ausgeprägt vorhanden. Da ein Erlernen neuer Fähigkeiten sehr zeitaufwändig und somit auch kostspielig sein kann, sollen die verwendeten Technologien auf den Kenntnissen des Entwicklerteams basieren und die bereits vorhandenen Fähigkeiten genutzt werden.
\newparagraph
Ein zusätzlicher wichtiger Faktor ist die weite Verbreitung der verwendeten Technologien. Je verbreiteter eine Technologie ist, desto einfacher ist es, für sie qualifiziertes Personal zu finden und Support zu erhalten. Außerdem besteht eine höhere Chance, dass die Technologie von mehreren anderen Unternehmen genutzt wird. Dadurch wird einer Abhängigkeit von einzelnen Marktteilnehmern entgegengewirkt.
\newparagraph
Eine hohe End-User-Kompatibilität ist ebenfalls wichtig, da jeder Kunde die Benutzeroberfläche einfach aufrufen können soll, ohne dass ein separates Setup auf seinem Endgerät erforderlich ist. Dies kann durch die Verwendung plattformunabhängiger Front-End-Technologien erreicht werden. Ein weiterer Vorteil einer plattformunabhängigen Entwicklung ist der Entwicklungsaufwand, der nur einmal durchgeführt werden muss, um ein für alle Plattformen ausgelegtes System zu schaffen und nicht für jede Plattform einzelne Entwicklungsaufwände betrieben werden müssen.
\newparagraph
Ein weiteres wichtiges Ausschlusskriterium für Technologien ist die Einhaltung gesetzlicher Grundlagen, insbesondere im Bereich Datenschutz. Dies ist besonders wichtig beim Scan von Rechnungen und anderen Dokumenten, die sensible persönliche Daten enthalten können. Es ist unerlässlich, dass jede verwendete Technologie die Anforderungen des Datenschutzgesetzes erfüllt und die Privatsphäre der Betroffenen schützt. Außerdem sind die gesetzlichen Grundlagen zu den Lizenzen verschiedener Technologien zu beachten, damit diese nicht verletzt werden und der FKW Software Solutions Group hierdurch kein Schaden entstehen kann.
\newparagraph
Ebenfalls wichtig bei der Auswahl einer Technologie ist ihre Skalierbarkeit. Durch die Skalierbarkeit wird festgelegt, wie gut sich die Technologie an ihre Umgebung anpasst, sodass beispielsweise bei einer stark ansteigenden Anzahl an Aufrufen weiterhin genug Leistung vorhanden ist, um die Anwendung ohne Einschränkungen betreiben zu können. Da keine Informationen zum Nutzungsverhalten im System des Auftraggebers vorliegen, muss das neue System an alle möglichen Nutzungsverhalten angepasst werden können, weswegen die Skalierbarkeit der Technologien ein wichtiger Punkt bei der Auswahl ist.
\newparagraph
Die gewählte Technologie muss sich durch einen hohen Reifegrad auszeichnen, damit die Wahrscheinlichkeit für grundsätzliche Änderungen während der Umsetzung reduziert wird, die durch mögliche Bugfixes in der zugrundeliegenden Technologie entstehen könnten.
Zusätzlich muss die Technologie eine gewisse Zukunftssicherheit aufweisen, damit eine Langlebigkeit des Produkts gewährleistet werden kann. Somit wird präventiv gegen einen erneuten Entwicklungsaufwand nach kurzer Zeit vorgegangen.
\newparagraph
Außerdem ist eine Modularität der Technologien von Relevanz, um die maximale Unabhängigkeit und Austauschbarkeit der Technologien zu ermöglichen. Dadurch wird der Anpassungsbedarf bei Austausch einer Technologie minimiert.

\subsection{Anwendung der Ausschlusskriterien}

  %%%%%%%%%%%%%%%%%%%%%%% Literaturverzeichnis %%%%%%%%%%%%%%%%%%%%%%%
  \phantomsection
\addcontentsline{toc}{section}{Literatur}
\printbibliography
\newpage


  %%%%%%%%%%%%%%%%%%%%%%%%%%%%%% Anhang %%%%%%%%%%%%%%%%%%%%%%%%%%%%%%
  \renewcommand{\thetable}{\Alph{section}.\arabic{table}}
  \renewcommand{\thefigure}{\Alph{section}.\arabic{figure}}
  \renewcommand{\thelstlisting}{\Alph{section}.\arabic{lstlisting}}
  \pagenumbering{Alph}

  \begin{appendix}
  \section{Anhang}
\end{appendix}
\end{document}