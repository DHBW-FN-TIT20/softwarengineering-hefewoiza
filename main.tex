%%%%%%%%%%%%%%%%%%%%%%%%%%%%%%%%%%%%%%%%%%%%%%%%%%%%%%%%%%%%%%%%%%%%
%%%           Vorlage für eine Ausarbeitung an der DHBW          %%%
%%%                                                              %%%
%%%      Bereiche die bearbeitet werden müssen werden durch      %%%
%%%      einen solchen Kommentarblock eingeleitet und enden      %%%
%%%      mit der nächsten Trennlinie.                            %%%
%%%                                                              %%%
%%%      In dieser Datei müssen folgende Bereiche bearbeitet     %%%
%%%      werden:                                                 %%%
%%%      - Angaben zur Arbeit                                    %%%
%%%      - EIGENE KAPITEL EINFÜGEN                               %%%
%%%                                                              %%%
%%%      Benötigte Seiten und Verzeichnisse können unter         %%%
%%%      "Einführung und Verzeichnisse" ein- bzw. auskommentiert %%%
%%%      werden.                                                 %%%
%%%                                                              %%%
%%%%%%%%%%%%%%%%%%%%%%%%%%%%%%%%%%%%%%%%%%%%%%%%%%%%%%%%%%%%%%%%%%%%

\documentclass[a4paper,12pt]{article}
\usepackage[left=2.5cm,right=2.5cm,top=2.5cm,bottom=2.5cm,includehead]{geometry}      % Einstellungen der Seitenränder
\usepackage[english, ngerman]{babel}                                                  % deutsche Silbentrennung
\usepackage[utf8]{inputenc}                                                           % Umlaute
\usepackage[T1]{fontenc}													                                    % Umlaute auch richtig ausgeben
\usepackage{newtxtext,newtxmath}                                                      % Font = Times New Roman
\usepackage{hyperref}
\usepackage[nottoc]{tocbibind}
\usepackage{fancyhdr}
\usepackage{setspace}
\usepackage[backend=bibtex, citestyle=authoryear, bibstyle=authoryear]{biblatex}      % Bibliothek für Zitate
\usepackage{csquotes}                                                                 % Zusatzpacket für Zitate
\usepackage{amsmath}                                                                  % Zurücksetzen der Tabellen- und Abbildungsnummerierung je Sektion
\usepackage[labelfont=bf,aboveskip=1mm]{caption}                                      % Bild- und Tabellenunterschrift (fett)
\usepackage[bottom,multiple,hang,marginal]{footmisc}                                  % Fußnoten [Ausrichtung unten, Trennung durch Seperator bei mehreren Fußnoten]
\usepackage{graphicx}  
\graphicspath{{./images/}}                                                            % Grafiken
\usepackage[dvipsnames]{xcolor}                                                       % Farbige Buchstaben
\usepackage{wrapfig}                                                                  % Bilder in Text integrieren
\usepackage{enumitem}                                                                 % Befehl setlist (Zeilenabstand für itemize Umgebung auf 1 setzen)
\usepackage{listings}                                                                 % Quelltexte
\definecolor{commentgreen}{RGB}{87,166,74}                                            % Kommentar-Farbe für Quellcode
\lstset{numbers=left, numberstyle=\tiny, numbersep=8pt, frame=single, framexleftmargin=15pt, breaklines=true, commentstyle=\color{commentgreen}}
\usepackage{tabularx}                                                                 % Tabellen
\usepackage{multirow}                                                                 % Mehrzeilige Tabelleneinträge
\usepackage[addtotoc]{abstract}                                                       % Abstract
\usepackage[nohyperlinks, printonlyused, withpage]{acronym}                           % Abkürzungen
\usepackage{dirtree}                                                                  % Ordnerstruktur (z.B. für Anhang)
\usepackage{float}
\usepackage{pdfpages}

%%%%%%%%%%%%%%%%%%%%%%%%%%%%%%%%%%%%%%%%%%%%%%%%%%%%%%%%%%%%%%%%%%%%
%%%                      Angaben zur Arbeit                      %%%
%%%%%%%%%%%%%%%%%%%%%%%%%%%%%%%%%%%%%%%%%%%%%%%%%%%%%%%%%%%%%%%%%%%%
\def\vFirmenlogoPfad{}                  %% relativer Pfad Bsp.: images/Firmenlogo.png
\def\vDHBWLogoPfad{images/DHBW_logo.jpg}                          %% relativer Pfad Bsp.: images/DHBW_logo.jpg
\def\vUnterschrift{}                    %% Pfad zu Bild mit Unterschrift (für digitale Abgabe) Bsp.: images/Unterschrift.png

\def\vTitel{Software Engineering 2}                           %% 
\def\vUntertitel{}                      %% 
\def\vArbeitstyp{Hausarbeit}                      %% Projektarbeit/Seminararbeit/Bachelorarbeit
\def\vArbeitsbezeichnung{}              %% T1000/T2000/T3000

\def\vLB{Lukas Braun}
\def\vJB{Johannes Brandenburger}
\def\vDF{David Felder}
\def\vFG{Florian Glaser}
\def\vFH{Florian Herkommer}
\def\vPP{Phillipp Patzelt}
\def\vHS{Henry Schuler}
\def\vBS{Baldur Siegel}



\def\vAutor{\vJB, \vLB, \vDF, \vFG, \vFH, \vPP, \vHS, \vBS}                           %% Vorname Nachname
\def\vMatrikelnummer{}                  %% 7-stellige Zahl
\def\vKursKuerzel{TIT20}                     %% Bsp.: TIT20
\def\vPhasenbezeichnung{Theoriephasen}               %% Praxisphase/Theoriephase
\def\vStudienJahr{dritte}                     %% erste/zweite/dritte
\def\vDHBWStandort{Ravensburg}                    %% Bsp.: Ravensburg
\def\vDHBWCampus{Friedrichshafen}                      %% Bsp.: Friedrichshafen
\def\vFakultaet{Technik}                       %% Technik/Wirtschaft
\def\vStudiengang{Informatik}                     %% Informationstechnik/...
\def\vKurs{TIT20}                     %% IT/...

\def\vBearbeitungsort{Friedrichshafen}                 %%                       %% 
\def\vBetreuer{Prof. Dr. Andreas Judt}                        %% Vorname Nachname

\def\vAbgabedatum{\today}               %% DD. MONTH YYYY
\def\vBearbeitungszeitraum{01.10.2022 - 23.12.2022}            %% DD.MM.YYYY - DD.MM.YYYY
%TODO Datum anpassen

%%%%%%%%%%%%%%%%%%%%%%%%% Eigene Kommandos %%%%%%%%%%%%%%%%%%%%%%%%%
% Definition von \gqq{}: Text in Anführungszeichen
\newcommand{\gqq}[1]{\glqq #1\grqq}
% Definition von \gq{}: Text in Anführungszeichen
\newcommand{\gq}[1]{\glq #1\grq}
% Spezielle Hervorhebung von Schlüsselwörtern
\newcommand{\textOrdner}[1]{\texttt{#1}}
\newcommand{\textVariable}[1]{\texttt{#1}}
\newcommand{\textKlasse}[1]{\texttt{#1}}
\newcommand{\textFunktion}[1]{\texttt{#1}}
\newcommand{\newparagraph}{\newline \newline}
% Quellenangabe bei Bildern
\newcommand{\customcaption}[2]{\caption[#1]{ #1. #2.}}
\def\vFKW{FKW Software Solutions}

%%%%%%%%%%%%%%%%%%%% Zitatbibliothek einbinden %%%%%%%%%%%%%%%%%%%%%
\addbibresource{./literatur/literatur.bib}


%%%%%%%%%%%%%%%%%%%%%%%% PDF-Einstellungen %%%%%%%%%%%%%%%%%%%%%%%%%
\hypersetup{
  bookmarksopen=false,
	bookmarksnumbered=true,
	bookmarksopenlevel=0,
  pdftitle=\vTitel,
  pdfsubject=\vTitel,
  pdfauthor=\vAutor,
  pdfborder={0 0 0},
	pdfstartview=Fit,
  pdfpagelayout=SinglePage
}


%%%%%%%%%%%%%%%%%%%%%%%% Kopf- und Fußzeile %%%%%%%%%%%%%%%%%%%%%%%%
\pagestyle{fancy}
\setlength{\headheight}{15pt}
\fancyhf{}
\fancyhead[R]{\thepage}


%%%%%%%%%%%%%%%%%%%%%%%%%%%%%% Layout %%%%%%%%%%%%%%%%%%%%%%%%%%%%%%
\onehalfspacing
\setlist{noitemsep}

\addto\captionsngerman{
  \renewcommand{\figurename}{Abb.}
  \renewcommand{\tablename}{Tab.}
}
\numberwithin{table}{section}                               % Tabellennummerierung je Sektion zurücksetzen
\numberwithin{figure}{section}                              % Abbildungsnummerierung je Sektion zurücksetzen
\renewcommand{\thetable}{\arabic{section}.\arabic{table}}   % Tabellennummerierung mit Section
\renewcommand{\thefigure}{\arabic{section}.\arabic{figure}} % Abbildungsnummerierung mit Section
\renewcommand{\thefootnote}{\arabic{footnote}}              % Sektionsbezeichnung von Fußnoten entfernen

\renewcommand{\multfootsep}{, }                             % Mehrere Fußnoten durch ", " trennen


%%%%%%%%%%%%%%%%%%%%%%%%%%%%% Dokument %%%%%%%%%%%%%%%%%%%%%%%%%%%%%

\begin{document}


  %%%%%%%%%%%%%%%%%%% Einführung und Verzeichnisse %%%%%%%%%%%%%%%%%%%
  \pagenumbering{Roman}

  \begin{titlepage}
  \begin{minipage}{6in}
    \vspace*{-2cm}
    \centering
    \hspace{-2cm}
	\ifx\vFirmenlogoPfad\empty
	\else
    \raisebox{-0.5\height}{\includegraphics[height=4cm]{\vFirmenlogoPfad}}
  \fi
	\hfill
	\ifx\vDHBWLogoPfad\empty
	\else
   	\raisebox{-0.5\height}{\includegraphics[height=4cm]{\vDHBWLogoPfad}}
	\fi
  \end{minipage}
  \begin{center}
    \vspace*{0.5cm}
    \Huge\textbf{\vTitel}\\
		\ifx\vUntertitel\empty
		\else
			\Large\rm\vUntertitel\\
		\fi
		\vspace*{2cm}
		\Large\textbf{\vArbeitstyp}
		\ifx\vArbeitsbezeichnung\empty
		\else
			\textbf{\vArbeitsbezeichnung}
		\fi
		\\
		\normalsize
		über die \vPhasenbezeichnung\ des \vStudienJahr{n}\ Studienjahrs \\
		\vspace*{1cm}
		an der Fakultät für \vFakultaet\\
		im Studiengang \vStudiengang\\
		\vspace*{0.5cm}
		an der DHBW \vDHBWStandort\\
		\ifx\vDHBWCampus\empty
		\else
		Campus \vDHBWCampus\\
		\fi
		\vspace*{0.5cm}
		von\\
		\ifx\vAutor\empty
		\else
			\vAutor\\
		\fi
		\vspace*{1cm}
		\vAbgabedatum
		\vfill
  \end{center}
  \begin{tabular}{ll}
    Bearbeitungszeitraum:&\vBearbeitungszeitraum\\
    Kurs:&\vKurs\\
	  Dozent der Hochschule:&\vBetreuer\\
  \end{tabular}
\end{titlepage}
\newpage
\setcounter{page}{2}
  % \thispagestyle{empty}
\section*{\Huge{Sperrvermerk}}

\addcontentsline{toc}{section}{Sperrvermerk}
gemäß Ziffer 1.1.13 der Anlage 1 zu §§ 3, 4 und 5  der Studien- und Prüfungsordnung für die Bachelorstudiengänge im Studienbereich Technik der Dualen Hochschule Baden-Würt­tem­berg vom 29.09.2017.\\

\noindent \gqq{Der Inhalt dieser Arbeit darf weder als Ganzes noch in Auszügen Personen außerhalb des Prüfungsprozesses und des Evaluationsverfahrens zugänglich gemacht werden, sofern keine anders lautende Genehmigung vom Dualen Partner vorliegt.}

\vfill
\leavevmode
\newline
\parbox{6cm}{\strut\centering \vBearbeitungsort, \vAbgabedatum\hrule\strut\centering\footnotesize Ort, Datum} 
\hfill
\ifx\vUnterschrift\empty
\parbox{6cm}{\strut\hspace{1pt} \vAbteilung\hrule\strut\centering\footnotesize Abteilung, Unterschrift}
\else
\parbox{6cm}{\strut\hspace{1pt} \vAbteilung, \parbox[b]{3cm}{\vspace{-10cm}\includegraphics[width=3cm]{\vUnterschrift}}\hrule\strut\centering\footnotesize Abteilung, Unterschrift}
\fi
\vspace{1cm}

\newpage
  \thispagestyle{empty}
\section*{\Huge{Gender Erklärung}}

\addcontentsline{toc}{section}{Gendererklärung}
Aus Gründen der besseren Lesbarkeit wird in dieser Bachelorarbeit auf die gleichzeitige Verwendung der Sprachformen männlich,
weiblich und divers (m/w/d) verzichtet. Sämtliche Formulierungen gelten gleichermaßen für alle Geschlechter.
\newpage
  \thispagestyle{empty}
\section*{\Huge{Selbstständigkeitserklärung}}

\addcontentsline{toc}{section}{Selbstständigkeitserklärung}
gemäß Ziffer 1.1.13 der Anlage 1 zu §§ 3, 4 und 5  der Studien- und Prüfungsordnung für die Bachelorstudiengänge im Studienbereich Technik der Dualen Hochschule Baden-Würt­tem­berg vom 29.09.2017.

\noindent Wir versichern hiermit, dass wir unsere Bachelorarbeit (bzw. Projektarbeit oder Studienarbeit bzw. Hausarbeit) mit dem Thema: 
\begin{center}
	\Large\textbf{\vTitel}
\end{center}
selbstständig verfasst und keine anderen als die angegebenen Quellen und Hilfsmittel benutzt haben. Wir versichern zudem, dass die eingereichte elektronische Fassung mit der gedruckten Fassung übereinstimmt.

\vfill
\leavevmode
\newline
\parbox{7cm}{\strut\centering \vBearbeitungsort, \vAbgabedatum\hrule\strut\centering\footnotesize Ort, Datum} 
\hfill
\parbox{7cm}{\strut\hspace{1pt} \hrule\strut\centering\footnotesize \vJB}
\newline
\vspace{1cm}
\newline
\parbox{7cm}{\strut\centering \vBearbeitungsort, \vAbgabedatum\hrule\strut\centering\footnotesize Ort, Datum} 
\hfill
\parbox{7cm}{\strut\hspace{1pt} \hrule\strut\centering\footnotesize \vLB}
\newline
\vspace{1cm}
\newline
\parbox{7cm}{\strut\centering \vBearbeitungsort, \vAbgabedatum\hrule\strut\centering\footnotesize Ort, Datum} 
\hfill
\parbox{7cm}{\strut\hspace{1pt} \hrule\strut\centering\footnotesize \vDF}
\newline
\vspace{1cm}
\newline
\parbox{7cm}{\strut\centering \vBearbeitungsort, \vAbgabedatum\hrule\strut\centering\footnotesize Ort, Datum} 
\hfill
\parbox{7cm}{\strut\hspace{1pt} \hrule\strut\centering\footnotesize \vFG}
\newline
\vspace{1cm}
\newline
\parbox{7cm}{\strut\centering \vBearbeitungsort, \vAbgabedatum\hrule\strut\centering\footnotesize Ort, Datum} 
\hfill
\parbox{7cm}{\strut\hspace{1pt} \hrule\strut\centering\footnotesize \vFH}
\newline
\vspace{1cm}
\newline
\parbox{7cm}{\strut\centering \vBearbeitungsort, \vAbgabedatum\hrule\strut\centering\footnotesize Ort, Datum} 
\hfill
\parbox{7cm}{\strut\hspace{1pt} \hrule\strut\centering\footnotesize \vPP}
\newline
\vspace{1cm}
\newline
\parbox{7cm}{\strut\centering \vBearbeitungsort, \vAbgabedatum\hrule\strut\centering\footnotesize Ort, Datum} 
\hfill
\parbox{7cm}{\strut\hspace{1pt} \hrule\strut\centering\footnotesize \vHS}
\newline
\vspace{1cm}
\newline
\parbox{7cm}{\strut\centering \vBearbeitungsort, \vAbgabedatum\hrule\strut\centering\footnotesize Ort, Datum} 
\hfill
\parbox{7cm}{\strut\hspace{1pt} \hrule\strut\centering\footnotesize \vBS}
\newpage
  %\phantomsection
\newenvironment{keywords}{
	\begin{flushleft}
	\small	
	\textbf{
		\iflanguage{ngerman}{Schlüsselwörter}{\iflanguage{english}{Keywords}{}}
	}
}{\end{flushleft}}

% Deutsche Zusammenfassung
\begin{abstract}
	
\end{abstract}

% Schlüsselwörter Deutsch
\begin{keywords}
	
\end{keywords}


\selectlanguage{english}
% Englisches Abstract
\begin{abstract}

\end{abstract}

% Schlüsselwörter Englisch
\begin{keywords}

\end{keywords}


\selectlanguage{ngerman}
\newpage
  \pdfbookmark[1]{\contentsname}{toc}
\tableofcontents
\newpage
  \section*{Abkürzungsverzeichnis}
\addcontentsline{toc}{section}{Abkürzungsverzeichnis}
\begin{acronym}
  \acro{DHBW}[DHBW]{Duale Hochschule Ba\-den-\-Würt\-tem\-berg}
  \acroplural{DHBW}[DHBW]{Dualen Hochschule Ba\-den-\-Würt\-tem\-berg}
  \acro{AWS}[AWS]{Amazon Web Services}
\end{acronym}
\newpage
  \listoffigures
\newpage
  \listoftables
\newpage
  \lstlistoflistings
\addcontentsline{toc}{section}{Listings}
\newpage
  % \section*{Vorwort}
\addcontentsline{toc}{section}{Vorwort}
\newpage


  %%%%%%%%%%%%%%%%%%%%%%%%%%%%% Kapitel %%%%%%%%%%%%%%%%%%%%%%%%%%%%%%
  \pagestyle{fancy}
  \fancyhead[L]{\nouppercase{\rightmark}}    % Abschnittsname im Header
  \pagenumbering{arabic}

  %%%%%%%%%%%%%%%%%%%%%%%%%%%%%%%%%%%%%%%%%%%%%%%%%%%%%%%%%%%%%%%%%%%%
  %%%%                   EIGENE KAPITEL EINFÜGEN                  %%%%
  %%%%%%%%%%%%%%%%%%%%%%%%%%%%%%%%%%%%%%%%%%%%%%%%%%%%%%%%%%%%%%%%%%%%
  \section{Systemidee}

Im Rahmen dieser \ac{OOA} soll das Softwaresystem \textit{\ac{aMRS}} konzeptioniert werden.
\newparagraph
Das Ziel des Systems ist die Verwirklichung der vom \hyperref[gls:auftraggeber]{Auftraggeber} gewünschten Erweiterung des \hyperref[gls:informationsportal]{Informationsportals} zum Angebot von Tagesgerichten um eine \hyperref[gls:authentischeBewertung]{authentische Bewertung} veganer Gerichte.
Hierzu schlagen wir die \textit{\vFKW}, das nachfolgend spezifizierte Softwaresystem \textit{\ac{aMRS}} vor, welches den Bewertungsprozess von veganen Speisen realisiert und somit die gestellten Anforderungen abdeckt.
\textit{\ac{aMRS}} verwaltet vegane Speisen des \hyperref[gls:informationsportal]{Informationsportals} und bietet Kunden die Möglichkeit, ihr Essen zu bewerten.
Diese Bewertungen können anschließend vom \hyperref[gls:informationsportal]{Informationsportal} abgerufen werden.
Um sicherzustellen, dass die Bewertungen authentisch sind, verifiziert das System den Kunden als tatsächlichen Käufer und überprüft die Bewertungen zusätzlich auf ihre Inhalte.
\newparagraph
Das von uns konzipierte System \textit{\ac{aMRS}} erhält die veganen Tagesgerichte vom \hyperref[gls:informationsportal]{Informationsportal} und speichert diese.
Somit wird die Bewertung der Gerichte zu einem späteren Zeitpunkt ermöglicht.
Erstellte Bewertungen können zusätzlich wiederkehrenden Tagesgerichten zugeordnet und bereitgestellt werden. Grundsätzlich werden sämtliche Daten persistent in \textit{\ac{aMRS}} gespeichert.
Dies macht \textit{\ac{aMRS}} erweiterbar und flexibel.
\newparagraph
Für den Bewertungsprozess bietet unsere Software-Lösung eine eigene Oberfläche, die parallel betrieben wird.
Somit müssen nur minimale (sonst teure) Anpassungen am bereits vorhandenen \hyperref[gls:informationsportal]{Informationsportal} getätigt werden.
Die Käufer-Verifizierung erfolgt ebenfalls unabhängig vom \hyperref[gls:informationsportal]{Informationsportals} des \hyperref[gls:auftraggeber]{Auftraggebers} über einen Scan der \hyperref[gls:restaurant]{Restaurant}-Rechnung.
Somit wird der Kauf verifiziert, ohne dass teure Anpassungen bei den \hyperref[gls:restaurant]{Restaurants} (z. B. ein Kassen-Plugin) nötig sind.

\subsection{Voraussetzungen an das Informationsportal}
Es wird vorausgesetzt, dass das System des \hyperref[gls:auftraggeber]{Auftraggebers} eine Schnittstelle bereitstellt, worüber die Daten zu den aktuellen Tagesgerichten abgerufen werden können.
Ein Gericht wird eindeutig durch einen Namen, ein Restaurantname, eine Restaurantadresse und einen \hyperref[gls:Rechnungsbezeichner]{Rechnungsbezeichner} identifiziert.
Der \hyperref[gls:Rechnungsbezeichner]{Rechnungsbezeichner} wird von den \hyperref[gls:restaurant]{Restaurants} definiert und ist identisch zu dem jeweiligen Bezeichner auf dem gedruckten Kassenbeleg.
Außerdem müssen die veganen Speisen als solche eindeutig gekennzeichnet sein.
\newparagraph
Um einen Bewertungsprozess zu starten, muss das \hyperref[gls:informationsportal]{Informationsportal} eine Verknüpfung bzw. einen Verweis zu \textit{\ac{aMRS}} ermöglichen.
Teilnehmende \hyperref[gls:restaurant]{Restaurants} müssen Rechnungen nach der aktuellen deutschen Gesetzeslage ausstellen.

\subsection{Kosten}

% Teamgröße: 8 Personen
% Stundensatz: 180 €
% Tage: 168 Manntage
% Stunden pro Tag: 8 h
% Gesamt: 241.920 €

Für die Entwicklung der ersten Version von \textit{\ac{aMRS}} werden in unserem 8-köpfigen Team 21 Arbeitstage benötigt.
Der Stundensatz wird hierbei auf 180~€ festgelegt.
Bei einem Arbeitstag mit 8 Arbeitsstunden ergibt dies eine Gesamtsumme von 241.920~€ (inkl. Steuern).




  %%%%%%%%%%%%%%%%%%%%%%% Literaturverzeichnis %%%%%%%%%%%%%%%%%%%%%%%
  \phantomsection
\addcontentsline{toc}{section}{Literatur}
\printbibliography
\newpage


  %%%%%%%%%%%%%%%%%%%%%%%%%%%%%% Anhang %%%%%%%%%%%%%%%%%%%%%%%%%%%%%%
  \renewcommand{\thetable}{\Alph{section}.\arabic{table}}
  \renewcommand{\thefigure}{\Alph{section}.\arabic{figure}}
  \renewcommand{\thelstlisting}{\Alph{section}.\arabic{lstlisting}}
  \pagenumbering{Alph}

  \begin{appendix}
  \section{Anhang}
\end{appendix}
\end{document}